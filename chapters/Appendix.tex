% Problem Definition (BA Ausschreibung) + Headings
\includepdf[trim=0cm 3cm 0cm 0cm,pages=-,pagecommand={\section{Project Management} \label{sec:Appendix Project Management} \subsection{Problem Definition} \label{problem_definition}},width=\textwidth]{ressources/Problem_Definition}

\subsection{Path Planning Calendar Week Plan} \label{sec:Appendix Path Planning Calendar Week Plan}
\begin{figure}[H]
    \centering
    \includegraphics[width=\columnwidth]{Weekly_Plan.png}
    \label{fig:Weekly Plan}
\end{figure}

\pagebreak

% Weekly Meetings Notes
\includepdf[pages=1,pagecommand={\subsection{Weekly Meetings Notes}  \label{sec:Appendix Weekly Meeting Notes}},width=\textwidth]{ressources/meeting_notes/notes_complete}
\includepdf[pages=2-,pagecommand={},width=\textwidth]{ressources/meeting_notes/notes_complete}

\section{Miscellaneous} \label{sec:Appendix Miscellaneous}

\subsection{ROS Path Planning Repository \newline Code Structure} \label{sec:Appendix Code Structure}
The source code of the ROS Path Planning Project is hosted on the GitHub Enterprise server of the \acrshort{zhaw}. \cite{ros_path_planning_github}
The code base is only accessible to members of \acrlong{zur}. Changes to the version control system were committed following the Conventional Commits specifications, and releases were numbered using Semantic Versioning 2.0.0. \cite{conventional_commits} \cite{semantic_versioning}
An overview of the code base is given in figure \ref{fig:Appendix Code Structure}.
\begin{figure}[H]
    \centering
    \includegraphics[width=7.20cm]{Appendix_Code_Structure.png}
    \caption{The structure of the ROS Path Planning Project, containing the source code of the Path Planning ROS Package.}
    \label{fig:Appendix Code Structure}
\end{figure}

The ``.vscode'' folder contains IDE-specific settings for Visual Studio Code. \cite{vscode_user_and_workspace_settings}
Setup files required by \acrshort{ros} are ``package.xml'' (containing meta information about the package), ``setup.cfg'' (required when a package has executables) and ``setup.py'' (containing instructions for how to install the package).

Inside the ``launch'' folder are all \acrshort{ros} launch files stored. Launch files provide a convenient way to start up multiple nodes and initialize requirements by setting parameters. \cite{clearpath_robotics_launch_files}

Inside the ``src'' folder are all \acrshort{ros} packages stored. That include all needed \acrshort{ros} messages inside the ``fszhaw\_msgs'' and the ``interfaces'' package, and the main Path Planning \acrshort{ros} package inside the ``path\_planning'' folder.
The source code of the Path Planning package is located inside another ``path\_planning'' folder. Resources like parameter files, tracks, and more are found in the ``resource'' folder. Test files are inside the ``test'' folder.

While development, temporary test files have been stored inside the ``\_testing'' folder, e.g. for testing functions, algorithms, and more.
The algorithms themselves are located inside the ``algorithm'' folder, which contains the Exploration and the Optimization Algorithm code.

The source code of the \acrshort{ros} nodes, which are only used for mocking external nodes, are found inside the ``mock'' folder. The Cone Publisher (``cone\_publisher.py'') mocks the Perception module and publishes the detected cones for the Path Planner. In contrast, the Planned Trajectory Subscriber (``planned\_trajectory.py'') receives the planned path outputted by the Path Planner.

Various models are found inside the ``model'' folder. These are only used for static typings and are optional for the functionality of the planner.
Utility classes, like the Optimization Input Transformer (``optimization\_input\_transformer.py'') and the Track Plotter (``track\_plotter.py'') are found inside the ``util'' folder. The Optimization Input Transformer transforms the input by the Exploration Algorithm to a format that the Optimization Algorithm can use. The Track Plotter can plot existing tracks for visual reference.

The main \acrshort{ros} nodes are on the level of the second ``path\_planning'' folder. ``path\_planner.py'' contains the main Path Planner node as described in section \ref{sec:Path Planner Node}, while ``optimization\_service.py'' contains the Optimization Service node as described in section \ref{sec:Optimization Service Node}.
Track-specific configurations, like the starting position or the distance thresholds, are found inside ``track\_config.py'' class.

\subsection{BA Path Planning Repository \newline Code Structure} \label{sec:Appendix BA Code Structure}
The source code of the BA Path Planning Project is hosted on the GitHub Enterprise server of the \acrshort{zhaw}. \cite{ros_path_planning_github}
The \Gls{latex} code is only accessible to the authors of the thesis. Changes to the version control system were committed following the Conventional Commits specifications, and releases were numbered using Semantic Versioning 2.0.0. \cite{conventional_commits} \cite{semantic_versioning}
An overview of the \Gls{latex} code is given in figure \ref{fig:Appendix BA Code Structure}.
\begin{figure}[H]
    \centering
    \includegraphics[width=7.20cm]{Appendix_BA_Code_Structure.png}
    \caption{The structure of the BA Path Planning Project, containing the Latex code of the thesis.}
    \label{fig:Appendix BA Code Structure}
\end{figure}

\pagebreak

The ``.vscode'' folder contains IDE-specific settings for Visual Studio Code. \cite{vscode_user_and_workspace_settings}
The preamble is found in ``preamble.tex'' and defines the document type, the language, loads extra packages, and sets several parameters. It contains all the information that occurs before the document begins.

The main entry point is found in ``main.tex''; this is where the actual document is defined. The content of the individual chapters are located inside separate Latex files (e.g., ``Abstract.tex'' containing the abstract or ``Background.tex'' containing the Background chapter) in the ``chapters'' folder.

Images are stored inside the ``img'' folder, while other resources like figures and meeting notes are stored inside the ``resources'' folder. The bibliography entries are defined inside the ``main.bib'' file. The entries of the glossary are found inside the ``glossary.tex'' file.

\pagebreak

\subsection{Optimization Algorithm Parameters} \label{sec:Appendix Optimization Algorithm Parameters}

\subsubsection{GGV Diagram} \label{sec:Appendix GGV Diagram}
The default supplied GGV diagram is shown in table \ref{tab:GGV Diagram}.
\begin{table}[H]
    \centering
    \begin{tabular}{|l|l|l|}
        \hline
        \textbf{v\_mps} & \textbf{ax\_max\_mps2} & \textbf{ay\_max\_mps2} \\ \hline
        0.0             & 12.0                   & 12.0                   \\ \hline
        4.0             & 12.0                   & 12.0                   \\ \hline
        8.0             & 12.0                   & 12.0                   \\ \hline
        12.0            & 12.0                   & 12.0                   \\ \hline
        16.0            & 12.0                   & 12.0                   \\ \hline
        20.0            & 12.0                   & 12.0                   \\ \hline
        24.0            & 12.0                   & 12.0                   \\ \hline
        28.0            & 12.0                   & 12.0                   \\ \hline
        32.0            & 12.0                   & 12.0                   \\ \hline
        36.0            & 12.0                   & 12.0                   \\ \hline
        40.0            & 12.0                   & 12.0                   \\ \hline
        44.0            & 12.0                   & 12.0                   \\ \hline
        48.0            & 12.0                   & 12.0                   \\ \hline
        52.0            & 12.0                   & 12.0                   \\ \hline
        56.0            & 12.0                   & 12.0                   \\ \hline
        60.0            & 12.0                   & 12.0                   \\ \hline
        64.0            & 12.0                   & 12.0                   \\ \hline
        68.0            & 12.0                   & 12.0                   \\ \hline
        72.0            & 12.0                   & 12.0                   \\ \hline
    \end{tabular}
    \caption{The supplied GGV diagram (``ggv.csv''), also known as the performance envelope, shows the maximum acceleration the car can sustain in any horizontal direction when travelling at any speed.}
    \label{tab:GGV Diagram}
\end{table}

\pagebreak

\subsubsection{Ax Machines Array} \label{sec:Appendix Ax Machines Array}
The default supplied ``Ax Machines Array'' is shown in table \ref{tab:Ax Machines Array}.
\begin{table}[H]
    \centering
    \begin{tabular}{|l|l|l|}
        \hline
        \textbf{v\_mps} & \textbf{ax\_max\_machines\_mps2} \\ \hline
        0.0             & 5.3                              \\ \hline
        4.0             & 5.3                              \\ \hline
        8.0             & 5.3                              \\ \hline
        12.0            & 5.3                              \\ \hline
        16.0            & 5.3                              \\ \hline
        20.0            & 5.3                              \\ \hline
        24.0            & 5.3                              \\ \hline
        28.0            & 5.3                              \\ \hline
        32.0            & 5.3                              \\ \hline
        36.0            & 5.3                              \\ \hline
        40.0            & 5.1                              \\ \hline
        44.0            & 5.0                              \\ \hline
        48.0            & 4.6                              \\ \hline
        52.0            & 4.1                              \\ \hline
        56.0            & 3.7                              \\ \hline
        60.0            & 2.7                              \\ \hline
        66.0            & 2.2                              \\ \hline
        72.0            & 1.5                              \\ \hline
    \end{tabular}
    \caption{The supplied ``ax\_max\_machines.csv'', containing an array with the longitudinal acceleration limits by the electrical motors.}
    \label{tab:Ax Machines Array}
\end{table}

\subsection{Scrum} \label{sec:Scrum}
Scrum is a framework which can be used by people, teams and organizations to solve problems through adaptive solutions. Four simple steps describe the process of Scrum which is lead by a Scrum Master:
\begin{enumerate}
    \item The Product Owner orders the work for a problem and puts it into a Product Backlog.
    \item The Scrum Team selects certain tasks from the Backlog, gives a number based on a certain expense and tries to realize it in a Sprint.
    \item The Scrum Team and stakeholders analyse the result and adjust it for the next sprint.
    \item After this the above steps will be repeated.
\end{enumerate}

These simplified steps help to explain how Scrum works. Scrum consists of several team roles which made up the Scrum Team. Scrum Events define what has to be done in each type of meeting. The figure \ref{fig:Scrum Framework} shows the Scrum Framework and gives an overview of the Events and Artefacts of Scrum. \cite{scrum_guide}


\begin{figure}[H]
    \centering
    \includegraphics[width=\columnwidth]{Scrumorg_Scrum_Framework.png}
    \caption{The Scrum Framework starts with the Product Backlog and will give through the Sprint Planning certain tasks to the Sprint Backlog. The developers in the Scrum Team work on the tasks in a Sprint and define the Increment of each Sprint. After the Sprint is finished a Sprint Review will be made. \cite{scrum_guide}}
    \label{fig:Scrum Framework}
\end{figure}


\subsubsection{Scrum Team} \label{sec:Scrum Team}
The Scrum Team consists of one Scrum Master, a Product Owner and Developers. Normally a Scrum Team has 10 or fewer people. There are other methods to plan like Scrum on a corporate or with large amount of peoples. If the team gets to large they should consider splitting the team in smaller ones. Bellow are all the team roles with the task their doing in a Scrum Team. \cite{scrum_guide}

\textbf{Developers} are the people that create the product. Several skills are needed to work as a developer. For the Scrum specific skills they have to create a plan for the Sprint which is called the Sprint Backlog. Furthermore, they have to create methods to maintain the quality to be sure what 'Done' means in a Sprint. Next they have to adapt their plan every day until the Sprint Goal is reached. One of the most important tasks for developers is to hold themselves accountable on a professional level. \cite{scrum_guide}

\textbf{Product Owner} is responsible for the Product Backlog. The tasks of a Product Owner consist of explicitly communication the Product Goal, to create and communicate each item in the Product Backlog, orders the Product Backlog items and checks if the Product Backlog is understood by every team member. \cite{scrum_guide} The Product Owner can be a working college or an employer from another firm.

\textbf{Scrum Master} is the role where the person ensures that all members are following the Scrum Guide. The theory and practices has to be understood by every member. Self-management and cross-functionality are the abilities that the Scrum Master coaches on the team. The Scrum Master helps the developers to define 'Done' in each Sprint and ensures that every Scrum Event is held within the time frame. In terms of the Product Owner the Scrum Master helps to define the Product Goal and Backlog and serves as a middle man between the Product Owner and the developers so that everything is understood. Furthermore, he collaborates with the stakeholders if there are any requests or needs. In the Organization a Scrum Master is responsible to get employers trained on Scrum, planning and advise the implementation, makes sure that everyone understand how Scrum works and removes walls between stakeholders and the Scrum Teams. \cite{scrum_guide}

\subsubsection{Scrum Events} \label{sec:Scrum Events}
Scrum has so-called Scrum Events that are minimizing the need for regular meetings and to focus on transparency. The events are as followed:
\textbf{The Sprint} is the foundation of Scrum. They are normally held once a month or less. The durance of a Sprint durance is as long until another Sprint starts. In a Sprint the Product Goal, Sprint Planning, Daily Scrums, the Sprint Review and the Retrospective will be defined. During the Sprint it is not allowed to make any changes on the Sprint Goal so that the quality is not decreasing. The Product Backlog can be refined, and the scope can be adjusted. \cite{scrum_guide}

\textbf{Sprint Planning} In the sprint planning the following questions will be answered: Why is this Sprint valuable? What can be done this Sprint? How will the chosen work get done? One of the roles should be able to answer the questions. \cite{scrum_guide}

\textbf{Daily Scrum} should be a 15-minute meeting where every developer provides the current status of his task. It helps to improve communication and helps in decision-making. \cite{scrum_guide}

\textbf{Sprint Review} covers the focus on the outcome of a sprint. The team shows the progress to the stakeholders. \cite{scrum_guide}

\textbf{Sprint Retrospective} helps to improve the quality of a Sprint. The Scrum Team critics what went wrong and find solutions to improve it. \cite{scrum_guide}

\subsubsection{Scrum Artefacts} \label{sec:Scrum Artefacts}

Artefacts define what work has to be done. To ensure transparency commitments are defined. The commitment for the Product Backlog is the Product Goal, for the Sprint Backlog it is the Sprint Goal and for the Increment it is the definition of 'Done'. Further explanations on the different artefacts are as followed: \cite{scrum_guide}

\textbf{Product Backlog} is a list of items that can be transferred to the Sprint Backlog. The items can be a task or a number of tasks that have to be done. Each item have a 'size' that define the hours which have to be invested to finish the task. \cite{scrum_guide}

\textbf{Sprint Backlog} is a list of items which have to be worked on for the current sprint to accomplish the Sprint Goal. \cite{scrum_guide}

\textbf{Increment} is a step to the Product Goal direction. The increments are named like that because they are additive to previous increments. The Definition of 'Done' defines the state of the Increment so that the requirements of that Increment is met. \cite{scrum_guide}