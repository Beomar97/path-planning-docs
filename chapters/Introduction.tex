\section{Initial situation}

\begin{itemize}
    \item Nennt bestehende Arbeiten/Literatur zum Thema
    \item Stand der Technik: Bisherige Lösungen des Problems und deren Grenzen
    \item «Stand der Technik» ist ein Fachbegriff, der den aktuellen Stand des Wissens im Thema meint. Sie beweisen damit, dass Sie das Fachgebiet kennen und das wesentliche Vorwissen aufgearbeitet haben.
    \item (Nennt kurz den Industriepartner und/oder weitere Kooperationspartner und dessen/deren Interesse am Thema Fragestellung)
\end{itemize}

When it comes to path planning algorithms, a lot of research has already been done. In the following chapter, a limited list of implementations is presented, with these algorithms being distinguished between past theses, state-of-the-art implementations in robotics and in Formula Student teams all around the world. This chapter also introduces the reader to the \acrlong{zur} (\acrshort{zur}) association of the \acrlong{zhaw}.

\subsection{Related Work}

Ignat Georgiev, a student at the University of Edinburgh, did a report on the topic "Path planning and control for an autonomous race car" in his master program. The report was based on the Formula Student project by the University of Edinburgh. The car follows the see-think-act cycle as an abstract way on how the path planning and control process is structured. By "see", he means what kind of information the sensors and cameras provide. With "think", the processing of the images, sensor data, mapping, localization and path planning is meant. And as for "act", the mechanism to steer, accelerate and break the vehicle is meant.

For path planning specifically, the report mentions two algorithms: the A* algorithm and the \acrlong{rrt} (\acrshort{rrt}) algorithm. In the DARPA Grand Challenge, all vehicles used a variety of the A* Graph Search Algorithm, with consideration of the kinematics. Furthermore, the incremental search algorithm \acrshort{rrt} was also used in the path in regard to autonomous driving.
To control the vehicle, another algorithm is used, the so-called \acrlong{mppi} (\acrshort{mppi}).
\cite{path_planning_and_control_georgiev}

\subsection{State of the Art}
This section gives a brief overview of different implementation of path planning algorithms. We differ between raw Python implementations and implementations by other Formula Student teams.

\subsubsection{Python Implementations}
PythonRobotics is a Python library hub for a wide range of different implementations in robotics. It varies between arm, aerial, car robotics and more.
For these segments of robotics, it provides an overview of different algorithm implementations, including several path planning algorithms.
The path planning algorithms differ from different approaches, dynamic window, breadth first search, Dijkstra algorithm, just to name a few. 
\cite{python_robotics}

\subsubsection{Formula Student Implementations}
The first student in the \acrshort{fszhaw} team to work on path planning was Jérôme Perdrizat. The implemented path planner consists of a middle line detection, path smoothening and a velocity / acceleration profile. The work was done as part of his master thesis. To evaluate the middle line of the track, Delaunay-Triangulation was used. An error rate was also mentioned in the thesis, e.g. when a cone wasn't detected. The thesis did not cover an algorithm which draws the optimum line in consideration of curves and speed.
\cite{autopilot_for_formula_student_jerome}

The \acrshort{amz} (\acrlong{amz}) from \acrshort{eth} (\acrlong{eth}) made Formula Student driverless resources available in a public repository, which also includes implementations of path planning algorithms. Several algorithms are being mentioned, e.g. the \acrshort{rrt} algorithm for track exploration and a \acrlong{mpcc} (\acrshort{mpcc}) for autonomous racing. Furthermore, for easier software distribution, a skeleton for building and implementing driverless algorithms is provided.
\cite{amz_racing_github}

% TODO Edinburgh and TUMFTM implementations

\subsection{Formula SAE}
In 1980, the idea of a combustion engine asphalt racing competition was born at the University of Texas by Ron Matthews. Initially it was called the SAE Mini Indy Competition, but was later renamed to \Gls{formula_sae}. Today, there are many events organized by local associations worldwide.
\cite{formula_sae}
These competitions challenge teams of university students to conceive, design, fabricate, develop, and compete with small, formula-style vehicles. They are held around the world and are one of the biggest student engineering competitions. As of today, there are approximately 600 teams. \cite{sae_student_events}

\subsubsection{Zurich UAS Racing}
Zurich UAS Racing (ZUR), formerly known as \acrlong{fszhaw} (\acrshort{fszhaw}), was launched back in 2019 with initially 20 members involved, with the first fully functional car finished in 2021. A new car is currently being built for the 2022 season, with currently more than 60 students involved. The new car has seen several improvements in aerodynamics, design, and most signifiant, with driverless capabilities. \cite{fszhaw_launch}
Therefore, for the first time in the association's history, the team will compete in the \acrshort{dc} category.
The ability to orientate itself in an unknown track and the ability to optimize a fully-known track are both critical components of a driverless race car. Therefore, it is imperative to research, implement and test various path planning algorithms. As part of the newly revised autonomous system, which also includes prediction, localization and controlling, a new path planning module has to be implemented.
The team will be competing at Formula Student events in Switzerland, organized by FSSwitzerland \cite{fsswitzerland}, Germany, organized by Formula Student Germany \cite{fs_germany}, and Croatia, organized by FS Alpe Adria \cite{fs_alpe_adria}.

\section{Problem Definition / Objective / Requirements}
\begin{itemize}
    \item Formuliert das Ziel der Arbeit
    \item Achtung: Ziel und Aufgabe sind nicht zwingend dasselbe! Bitte sauber trennen.
    \item Verweist auf die offizielle Aufgabenstellung des/der Dozierenden im Anhang
    \item (Pflichtenheft, Spezifikation)
    \item Spezifiziert die Anforderungen an das Resultat der Arbeit
    \item Übersicht über die Arbeit: stellt die folgenden Teile der Arbeit kurz vor
    \item Das erleichtert die Leserführung und schafft Klarheit.
    \item (Angaben zum Zielpublikum: nennt das für die Arbeit vorausgesetzte Wissen)
    \item (Terminologie: Definiert die in der Arbeit verwendeten Begriffe)
    \item Nur spezielle Fachbegriffe; man kann in der Regel von einem informierten Zielpublikum ausgehen.
    Wenn ein Glossar (vgl. 6.2.) erstellt wird, erübrigt sich dieser Abschnitt.
\end{itemize}

\subsection{Problem Definition}
\acrlong{zur}, the university's racing team, have begun their work on its second generation race car since Fall 2021. With path planning being an integral part of an autonomous car, the research and implementation of a new path planning module as part of their new autonomous system is very important. See the official problem definition in the appendix.

\subsection{Objective}
The objective of this thesis is the research and implementation of several path planning algorithms, which can be adapted on different tracks and driving scenarios. The end goal is to be able to compete in the \acrlong{dc} at various Formula Student events from July until August.

% TODO
% Übersicht über die Arbeit: stellt die folgenden Teile der Arbeit kurz vor
