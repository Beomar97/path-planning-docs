\section{Initial situation}
\begin{itemize}
    \item Nennt bestehende Arbeiten/Literatur zum Thema
    \item Stand der Technik: Bisherige Lösungen des Problems und deren Grenzen
    \item «Stand der Technik» ist ein Fachbegriff, der den aktuellen Stand des Wissens im Thema meint. Sie beweisen damit, dass Sie das Fachgebiet kennen und das wesentliche Vorwissen aufgearbeitet haben.
    \item (Nennt kurz den Industriepartner und/oder weitere Kooperationspartner und dessen/deren Interesse am Thema Fragestellung)
\end{itemize}

\lipsum[1] \cite{quelle2}

\section{Objective / Problem definition / Requirements}
\begin{itemize}
    \item Formuliert das Ziel der Arbeit
    \item Achtung: Ziel und Aufgabe sind nicht zwingend dasselbe! Bitte sauber trennen.
    \item Verweist auf die offizielle Aufgabenstellung des/der Dozierenden im Anhang
    \item (Pflichtenheft, Spezifikation)
    \item Spezifiziert die Anforderungen an das Resultat der Arbeit
    \item Übersicht über die Arbeit: stellt die folgenden Teile der Arbeit kurz vor
    \item Das erleichtert die Leserführung und schafft Klarheit.
    \item (Angaben zum Zielpublikum: nennt das für die Arbeit vorausgesetzte Wissen)
    \item (Terminologie: Definiert die in der Arbeit verwendeten Begriffe)
    \item Nur spezielle Fachbegriffe; man kann in der Regel von einem informierten Zielpublikum ausgehen.
    Wenn ein Glossar (vgl. 6.2.) erstellt wird, erübrigt sich dieser Abschnitt.
\end{itemize}

\lipsum[1] \cite{quelle2}

\subsection{Subsection}
\lipsum[1] \cite{quelle1}

\begin{figure}[H]
\centering
\includegraphics[width=0.4\columnwidth]{de-zhaw-init-rgb}
\caption{Bildli}
\label{fig:bildli1}
\end{figure}


\subsubsection{SubSubSection}
\lipsum[1]
\begin{table}[H]
\centering
\caption{Eine Tabelle}
\label{tab:my-table}
\begin{tabular}{|l|l|l|}
\hline
\textbf{A} & \textbf{B} & \textbf{C} \\ \hline
1          & 2          & 3          \\ \hline
4          & 5          & 6          \\ \hline
\end{tabular}
\end{table}

\paragraph{Paragraph}
\lipsum[1]

 


