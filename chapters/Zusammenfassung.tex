Formula Student ist ein weltweiter Ingenieur Wettbewerb, welcher an verschiedenen Orten auf der ganzen Welt ausgetragen wird. Das Ziel jedes Teams ist es, ein Rennauto von neu auf zu entwickeln und mit diesem am Renntag gegen andere Teams anzutreten. Es gibt Verbrennungs- und Elektromotor angetriebene Rennautos, welche in ihren eigenen Kategorien gegeneinander antreten. Das \acrlong{zur} Team der Zürcher Hochschule für angewandte Wissenschaften hat ein elektrogetriebenes Rennauto mit fahrerlosen Kapazitäten entwickelt. Mit den erfassten Pylonen am Rande der Strecke, welche die Strecke begrenzen, und der Position des Rennautos kommt ein Pfadplanungsalgorithmus zum Zuge, welches dem Rennauto erlaubt, zwischen den Pylonen zu manövrieren. Diese wissenschaftliche Arbeit stellt zwei Algorithmen vor, welche dieses Problem lösen. Der erste Algorithmus, der sogenannte Erkundungsalgorithmus, berechnet die Mittellinie der Strecke und übergibt diese dem zweiten Algorithmus, dem Optimierungsalgorithmus. Der zweite Algorithmus optimiert die Mittellinie zu einer Ideallinie. Dies geschieht zusätzlich mit der Beeinflussung der Autoeckdaten. Der Erkundungsalgorithmus kann für die sogenannte ``Acceleration'' Strecke (welche eine Gerade abbildet) und für unbekannte Strecken angewendet werden. Eine Erweiterung des Erkundungsalgorithmus für die ``Skidpad'' Strecke müsste noch realisiert werden. Die ``Skidpad'' Strecke ist eine Strecke, welche eine liegende Acht repräsentiert. Dieses Problem kann mit einer statischen Implementation, welches die vorausgespeicherten Koordinaten der Mittellinie nutzt, gelöst werden. Der Optimierungsalgorithmus funktioniert gut und effizient in Kombination mit der minimaler Krümmungzielsetzung, führt aber nicht zu den besten Zeiten im Vergleich mit der minimaler Zeitzielsetzung. Ein Wechsel zur minimaler Zeitzielsetzung könnte die geschätzten Rundenzeiten weiter verbessern, sobald alle notwendigen Parametern bereitgestellt werden können. Das Optimieren der Strecke in Teilstrecken während der Erkundungsphase könnte für die längeren Berechnungszeiten für diese Zielsetzung kompensieren.
