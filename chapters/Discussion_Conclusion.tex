This chapter interprets and validates the results from the tests on the Exploration and Optimization Algorithms in chapter \ref{ch:Results}. First, the problem definition will be revisited, and it will be discussed if the mentioned objectives have been achieved. Secondly, the results of the Exploration Algorithm are compared side by side and a conclusion is made, followed by the interpretation of the results with the Optimization Algorithm.

Back in section \ref{sec:Problem Definition / Objective / Requirements}, the research and implementation of a new path planning module as part of the new autonomous system of \acrlong{zur} has been defined as the problem to solve as the objective is to implement path planning algorithms that can adapt to different tracks and driving scenarios. The goal of researching and implementing several path planning algorithms has been achieved by implementing both the Exploration Algorithm, which can calculate the middle line of a track and the Optimization Algorithm, which then optimizes the calculated middle line to an optimal racing line. Both algorithms have proven usable on many different tracks and driving scenarios, like the Acceleration and Skidpad scenarios and several tracks like ``Small Track'', ``Rand'' or ``Comp 2021''.

\section{Exploration Algorithm}
The discussion and conclusion of the Exploration Algorithm are described in the following section. Foremost, the result of the acceleration track is interpreted. After that, several Track Drive tracks are compared and discussed. With these comparisons, the outcomes on the different tracks and the versatility of the algorithm can be explained. Lastly, the ``Skidpad'' track is interpreted based on the applied final implementation of the Exploration Algorithm.

Beginning with the interpretation of the acceleration track shown in section \ref{sec:Results Acceleration Track}. At the first look, every algorithm version can be used to calculate the middle line of the acceleration track. The advantage of the second implementation compared to the first one is in the additional points and smoothening of the path, which makes the results more precise and gives the car an improved path. The final implementation of the algorithm takes the approach of the second one further and creates additional points from the smoothened path to be even more precise so that the car can drive more accurately. Since the track is the most simplistic one mentioned, no more improvement could be made.

The second track was the ``Small Track'' which is illustrated in section \ref{sec:Results Small Track}. The track has a curvature, making it more complicated for the algorithm to handle. As seen in the comparison on the image of the first, second and latest algorithm implementation, the curve on the bottom left of the track shows a considerable difference between the algorithm versions. The curvature for the first and second approaches is too near the yellow cones, which is not feasible in a natural environment. An additional consideration of the difference between the calculated path and the actual path has to be made. If this difference means the delta is too big, the car will touch the cones. To conclude, only the last implementation is usable for a natural environment since it handles the curves in an improved way.

The section \ref{sec:Results Rand Track} and \ref{sec:Results Competition 2021 Track} provides an illustration of a possible track for the third discipline in the competition, which is the unknown track. The track of the 2021 competition provides a tangible example of how the Exploration Algorithm can be applied to get to the middle line in the first lap. For both tracks, the final implementation of the exploration algorithm is shown and provides evidence that it is feasible to test on a real race car.

The last track shows the final implementation of the exploration algorithm on the ``Skidpad'' track, which is described in section \ref{sec:Results Skidpad Track}. As shown in the illustration, the algorithm can be used for the ``Skidpad'' track, but further improvements to the curvature have to be made so that the algorithm accomplishes two rounds on the right circle and two rounds on the left circle before crossing the finish line. Since every track measurement is given in the rules book, static implementation of the middle line could be made as well, where the car would follow the manually created track that is saved in a CSV file, for example.

To conclude, the final implementation of the Exploration Algorithm can be used for various tracks and can handle narrower curvatures and wider ones. Further testing has to be made on the actual car in combination with the actual coordinates of the cones and localization of the car. A further improvement for predicting the route could calculate the middle line faster so that the car is less dependent on the speed, meaning that it could flood the ``path\_planner'' node with published cones. Different cone publisher rates were used in the testing phase, which did not influence the algorithm to the point where it could not calculate the middle line on time.

\section{Optimization Algorithm}
The Optimization Algorithm with the Minimum Curvature objective works well on all the tested tracks. Only on one occasion, on the Garden Light track, did the objective score a slower estimated lap time than the most straightforward objective, the Shortest Path objective, as detailed in section \ref{sec:Results Garden Light Track}. Accordingly, the Minimum Curvature objective proved to be the most efficient objective out of the available options in the Optimization Algorithm. While the Shortest Path objective was, on average, 25.05\% faster in computation than the Minimum Curvature objective, the estimated lap times were, on average, 54.71\% slower than the times with the Minimum Curvature objective. For the IQP Handler, the Minimum Curvature objective with the iterative call, the estimated lap times were 4.80\% faster than the Minimum Curvature objective without an iterative call. However, on average, its computation time was 107.24\% longer than the objective without an iterative call. Additionally, the handler did not finish in a reasonable time on two tracks, the Competition 2021 track and the Garden Light track, for reasons unknown, making the IQP Handler impractical. The comparison of the run- and estimated lap times of the Shortest Path objective and IQP Handler with the Minimum Curvature objective are detailed in table \ref{tab:Results Optimization Objectives Average}.

While the use of the IQP Handler makes it impractical because of its long computation times and not finishing results on some tracks, the Minimum Time objective shows faster computation times than the IQP Handler. Further, it outperforms both Minimum Curvature objectives with and without the iterative call. The only problem with the Minimum Time objective is its vast number of required parameters, ranging from the energy consumption limit of the vehicle to the maximal drive and brake force of the vehicle. Its still long computation time can be resolved by optimizing the explored track in legs instead of optimizing the whole track.

\begin{table}[H]
    \centering
    \begin{tabular}{|lll|}
        \hline
        \multicolumn{1}{|l|}{\textbf{Objective}}    & \multicolumn{1}{l|}{\textbf{Runtime}} & \multicolumn{1}{l|}{\textbf{Estimated Lap Time}} \\ \hline
        \multicolumn{3}{|l|}{\textbf{Small Track}}                                                                                             \\ \hline
        \multicolumn{1}{|l|}{Shortest Path}         & \multicolumn{1}{l|}{- 12.50\%}        & \multicolumn{1}{l|}{+ 11.47\%}                   \\ \hline
        \multicolumn{1}{|l|}{Minimum Curvature IQP} & \multicolumn{1}{l|}{+ 25.00\%}        & \multicolumn{1}{l|}{- 3.25\%}                    \\ \hline
        \multicolumn{3}{|l|}{\textbf{Rand}}                                                                                                    \\ \hline
        \multicolumn{1}{|l|}{Shortest Path}         & \multicolumn{1}{l|}{- 24.32\%}        & \multicolumn{1}{l|}{+ 22.10\%}                   \\ \hline
        \multicolumn{1}{|l|}{Minimum Curvature IQP} & \multicolumn{1}{l|}{+ 121.62\%}       & \multicolumn{1}{l|}{- 4.07\%}                    \\ \hline
        \multicolumn{3}{|l|}{\textbf{Comp 2021}}                                                                                               \\ \hline
        \multicolumn{1}{|l|}{Shortest Path}         & \multicolumn{1}{l|}{- 16.67\%}        & \multicolumn{1}{l|}{+ 0.71\%}                    \\ \hline
        \multicolumn{1}{|l|}{Minimum Curvature IQP} & \multicolumn{1}{l|}{DNF*}             & \multicolumn{1}{l|}{DNF*}                        \\ \hline
        \multicolumn{3}{|l|}{\textbf{Garden Light}}                                                                                            \\ \hline
        \multicolumn{1}{|l|}{Shortest Path}         & \multicolumn{1}{l|}{- 11.76\%}        & \multicolumn{1}{l|}{- 2.73\%}                    \\ \hline
        \multicolumn{1}{|l|}{Minimum Curvature IQP} & \multicolumn{1}{l|}{DNF*}             & \multicolumn{1}{l|}{DNF*}                        \\ \hline
        \multicolumn{3}{|l|}{\textbf{Rounded Rectangle}}                                                                                       \\ \hline
        \multicolumn{1}{|l|}{Shortest Path}         & \multicolumn{1}{l|}{- 17.95\%}        & \multicolumn{1}{l|}{+ 6.11\%}                    \\ \hline
        \multicolumn{1}{|l|}{Minimum Curvature IQP} & \multicolumn{1}{l|}{+ 51.28\%}        & \multicolumn{1}{l|}{- 13.54\%}                   \\ \hline
        \multicolumn{3}{|l|}{\textbf{Handling Track}}                                                                                          \\ \hline
        \multicolumn{1}{|l|}{Shortest Path}         & \multicolumn{1}{l|}{- 33.14\%}        & \multicolumn{1}{l|}{+ 3.96\%}                    \\ \hline
        \multicolumn{1}{|l|}{Minimum Curvature IQP} & \multicolumn{1}{l|}{+ 108.72\%}       & \multicolumn{1}{l|}{- 5.55\%}                    \\ \hline
        \multicolumn{3}{|l|}{\textbf{Berlin 2018}}                                                                                             \\ \hline
        \multicolumn{1}{|l|}{Shortest Path}         & \multicolumn{1}{l|}{- 39.87\%}        & \multicolumn{1}{l|}{+ 5.60\%}                    \\ \hline
        \multicolumn{1}{|l|}{Minimum Curvature IQP} & \multicolumn{1}{l|}{+ 168.56\%}       & \multicolumn{1}{l|}{- 1.15\%}                    \\ \hline
        \multicolumn{3}{|l|}{\textbf{Modena 2019}}                                                                                             \\ \hline
        \multicolumn{1}{|l|}{Shortest Path}         & \multicolumn{1}{l|}{- 44.19\%}        & \multicolumn{1}{l|}{+ 7.49\%}                    \\ \hline
        \multicolumn{1}{|l|}{Minimum Curvature IQP} & \multicolumn{1}{l|}{+ 168.27\%}       & \multicolumn{1}{l|}{- 1.22\%}                    \\ \hline
        \multicolumn{3}{|l|}{\textbf{Average}}                                                                                                 \\ \hline
        \multicolumn{1}{|l|}{Shortest Path}         & \multicolumn{1}{l|}{- 25.05\%}        & \multicolumn{1}{l|}{+ 54.71\%}                   \\ \hline
        \multicolumn{1}{|l|}{Minimum Curvature IQP} & \multicolumn{1}{l|}{+ 4.80\%}         & \multicolumn{1}{l|}{- 107.24\%}                  \\ \hline
    \end{tabular}
    \caption{The runtimes and estimated lap times of the Shortest Path objective and Minimum Curvature objective with an iterative call compared to the Minimum Curvature objective without an iterative call. (*Did not finish)}
    \label{tab:Results Optimization Objectives Average}
\end{table}