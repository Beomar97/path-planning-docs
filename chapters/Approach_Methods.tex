\begin{itemize}
    \item Beschreibt die Grundüberlegungen der realisierten Lösung (Konstruktion/Entwurf) und die Realisierung als Simulation, als Prototyp oder als Software-Komponente etc.
    \item Hier beschreiben Sie Ihre gemachte Arbeit. Dazu braucht es eine Beschreibung des Vorgehens, aller Arbeitsschritte usw.
    \item (Definiert Messgrössen, beschreibt Mess- oder Versuchsaufbau, beschreibt und dokumentiert Durchführung der Messungen/Versuche)
    \item Bildmaterial erleichtert das Verständnis.
    \item (Experimente)
    \item Immer mit Aufbau und Vorgehen; Bildmaterial erleichtert das Verständnis.
    \item (Lösungsweg)
    \item Inkl. theoretische Herleitung der Lösung
    \item (Modell)
    \item (Eingesetzte Software)
    \item Die Funktionen von verwendeten Computerprogrammen zu Simulationszwecken, Berechnungen etc. sollen beschrieben werden. Dies soll aber in Worten, Formeln und geeigneten Darstellungen (z.B. Fluss- diagrammen) geschehen. Allfälliger Programmcode ist in einem Anhang zu dokumentieren.
    \item (Tests und Validierung)
\end{itemize}

This chapter describes which approaches and methods have been used to solve the problem to find algorithms which calculate the optimum path on a racetrack.

\section{Marco's Proposal}
\begin{itemize}
    \item Grundüberlegung => Path Planning Package in ROS, Prototyp Architektur zeigen, Explo und Opt Algos, erhaltet Input und sendet Input
    \item Vorgehensmethode => Kanban und Scrum, maybe V-Model?
    \item Setup? Weekly meetings, Review every other week, Mitarbeit mit anderen BA Teams in Driverless und gesamt Verein (Working Saturday, Hilfe im Workshop, Ausstellung Conecto ZHAW)
    \item Setup technisch, lokale Linux VM, ROS Installation, VS Code als IDE, GitHub Repos, GitHub Actions CI/CD (maybe?), Deployment Architektur
    \item Overview Architektur ROS und Code
    \item Path Planner Node
    \item Optimization Service Node
    \item Messages (interfaces und fszhaw msgs)
    \item Exploration Algorithm
    \item erste Überlegungen und Tests, alter path planner zhaw, dann densify und interpolate, rrt max hamburg, komplexer algo à la ultimate und dann jetzige implementation
    \item Optimization Algorithm
    \item zuerst was hat eth mit mpcc, dann global racetrajectory von tumftm
    \item Testing mit Maps, Cone Publisher und Planned Trajectory Subscriber (mocks), utils wie track plotter, trackconfig
\end{itemize}

\section{Architecture Design} \label{sec:Architecture Design}
\begin{itemize}
    \item Beschreibt Architektur von ROS
    \item UML Klassendiagram
    \item Interfaces für Messages
    \item Kommunikationsdiagramm (zwischen Algorithmen)
\end{itemize}

\section{Exploration Algorithm} \label{sec:Exploration Algorithm}
To find the middle line an algorithm has to be used. The exploration algorithm finds the line on which the car should drive on the first round on the track. Exploration is possible with the help of the current position of the car and the cones which are recognized by the Lidar sensor and stereo camera while the car is moving along the track. The \acrshort{ros} Node ``/path\_planner'' subscribes on the ``cone'' and ``current\_position'' topic where it gets all the information needed to calculate the path. For the subscription of ``/current\_position'' a callback function named ``self.\_\_current\_position\_listener\_callback'' will save the position in a local variable. The ``cone'' subscription listens to new cones that will be published via the ``cone\_publisher'' Node.

\section{Optimization Algorithm} \label{sec:Optimization Algorithm}
After the explore algorithm analysed the middle line of the track the board computer switches to the optimization algorithm. For the base code of the algorithm we forked the code from \acrlong{tum} (\acrshort{tum}). \cite{tumftm_optimization_algoritm} 



\section{V-Model}
\lipsum[1]