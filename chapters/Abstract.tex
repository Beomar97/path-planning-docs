Formula student is a worldwide engineering competition held in different region across the globe. The target of every team is to develop a race car from scratch to compete against other teams on the race day. There are combustion and electric engine powered vehicles which have different subcategories of competition. The \acrlong{zur} team from the \acrlong{zhaw} has an electric powered race car. There is the challenge with the race car driver in the car and a challenge without the driver, named driverless. The driverless category has to accomplish to finish three different tracks without a human controlling the car. Two of three tracks are given by their dimensions and start position of the car. With the detected cones on the track and the position of the car a path planning algorithm has to be implemented, so that the car can maneuver between the cones. This thesis gives two examples of algorithms that can be used to solve this problem. The first one is the exploration algorithm which calculates the middle line and gives the data to the second algorithm that optimizes the middle line with the influence of the attributes of the car meaning the measurements, weight etc. The exploration algorithm accomplished different tracks like the ``Acceleration'' track (a straight line) and tracks that could be unknown to the team. Further improvements can be made on the ``Skidpad'' track which is a laying eight on that the car has to drive two times on the right circle and two times on the left one. This could be solved with a static implementation where the coordination of the track can be saved manually.
