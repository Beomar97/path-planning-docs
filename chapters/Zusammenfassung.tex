Formula Student ist ein weltweiter Ingenieur Wettbewerb, welcher an verschiedenen Orten auf der ganzen Welt ausgetragen wird. Das Ziel jedes Teams ist es, ein Rennauto von neu auf zu entwickeln und mit diesem am Renntag gegen andere Teams anzutreten. Es gibt Verbrennungs- und Elektromotor angetriebene Rennautos, welche in ihren eigenen Kategorien sich gegeneinander stellen. Das \acrlong{zur} Team der Zürcher Hochschule für angewandte Wissenschaften hat ein elektrogetriebenes Rennauto entwickelt, das ohne Fahrer sich auf einer gegebenen Strecke fortbewegen kann. Mit den von den Sensoren erfassten Pylonen als Streckenbegrenzung, und der Position des Rennautos, kommt ein Pfadplanungsalgorithmus zum Zuge. Der Algorithmus erlaubt dem Rennauto, zwischen den Pylonen sich zu manövrieren. Diese wissenschaftliche Arbeit stellt zwei Algorithmen vor, welche dieses Problem lösen. Der erste Algorithmus, der sogenannte Erkundungsalgorithmus, berechnet die Mittellinie der Strecke. Dieser kann für die sogenannte ``Acceleration'' Strecke und für verschiedene unbekannte Strecken angewendet werden. Eine Erweiterung des Erkundungsalgorithmus für die ``Skidpad'' Strecke müsste noch realisiert werden. Dieses Problem kann mit einer statischen Implementation, welches die vorausgespeicherten Koordinaten der Mittellinie nutzt, gelöst werden. Das Resultat des Erkundungsalgorithmus wird am zweiten Algorithmus, dem Optimierungsalgorithmus übergeben. Der zweite Algorithmus optimiert die Mittellinie zu einer Ideallinie. Dies geschieht zusätzlich mit der Beeinflussung der Autoeckdaten. Der Optimierungsalgorithmus funktioniert gut und effizient in Kombination mit der minimaler Krümmungszielsetzung, führt aber nicht zu den besten Zeiten im Vergleich mit der minimaler Zeitzielsetzung. Ein Wechsel zur minimaler Zeitzielsetzung könnte die geschätzten Rundenzeiten weiter verbessern, sobald alle notwendigen Parametern bereitgestellt werden können. Das Optimieren der Strecke in Teilstrecken während der Erkundungsphase, kann die längeren Berechnungszeiten für diese Zielsetzung kompensieren.
