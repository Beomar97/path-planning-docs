Formula Student ist ein weltweiter Ingenieur Wettbewerb, welcher an verschiedenen Orten auf der ganzen Welt ausgetragen wird. Das Ziel jedes Teams ist es, ein Rennauto von neu auf zu entwickeln und mit diesem am Renntag gegen andere Teams anzutreten. Dabei werden die Wettkämpfe in unterschiedliche Kategorien unterteilt, einerseits nach Antriebsart, Verbrennungs- und Elektromotor, und anderseits nach Rennen mit und ohne Fahrer. Das \acrlong{zur} Team der Zürcher Hochschule für angewandte Wissenschaften hat ein elektrisch angetriebenes Rennauto entwickelt. Für die automatisierte Fahrfunktion ohne Fahrer bedarf es neben der Erkennung der Pylonen der Streckenbegrenzung und der Lokalisierung des Fahrzeugs auf der Strecke, die Plannung des zu fahrenden Pfades. Dieser Pfad ist dabei so konzipiert, dass das Fahrzeug so sicher und schnell wie möglich durch die Strecke manövriert werden kann. Diese wissenschaftliche Arbeit stellt zwei Algorithmen vor, welche dieses Problem lösen. Der erste Algorithmus, der sogenannte Erkundungsalgorithmus, berechnet die Mittellinie der Strecke. Dieser kann für die sogenannte ``Acceleration'' Strecke und für verschiedene unbekannte Strecken angewendet werden. Eine Erweiterung des Erkundungsalgorithmus für die ``Skidpad'' Strecke müsste noch realisiert werden. Dieses Problem kann mit einer statischen Implementation, welches die vorausgespeicherten Koordinaten der Mittellinie nutzt, gelöst werden. Das Resultat des Erkundungsalgorithmus wird am zweiten Algorithmus, dem Optimierungsalgorithmus übergeben. Der zweite Algorithmus optimiert die Mittellinie zu einer Ideallinie. Dabei werden je nach Optimierungskriterium verschiedene Fahrzeugparameter berücksichtigt. Der Optimierungsalgorithmus liefert dabei bereits gute Resultate, wenn als Kriterium die minimale Krümmung des Pfades zugrunde gelegt wird. Eine weitere Verbesserung der Rennzeit könnte zudem erreicht werden, wenn diese Zeit als weiteres Optimierungskriterium angesetzt wird, wofür allerdings die Fahrzeugparameter genau bekannt sein müssen. Ein Wechsel zur minimaler Zeitzielsetzung könnte die geschätzten Rundenzeiten weiter verbessern, sobald alle notwendigen Parametern bereitgestellt werden können. Um die Berechnungszeiten zu reduzieren, könnte die Optimierung des Pfades für einzelne Teilstrecken bereits während der Erkundungsphase erfolgen.
