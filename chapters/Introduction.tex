\section{Initial situation}

\begin{itemize}
    \item Nennt bestehende Arbeiten/Literatur zum Thema
    \item Stand der Technik: Bisherige Lösungen des Problems und deren Grenzen
    \item «Stand der Technik» ist ein Fachbegriff, der den aktuellen Stand des Wissens im Thema meint. Sie beweisen damit, dass Sie das Fachgebiet kennen und das wesentliche Vorwissen aufgearbeitet haben.
    \item (Nennt kurz den Industriepartner und/oder weitere Kooperationspartner und dessen/deren Interesse am Thema Fragestellung)
\end{itemize}

%\lipsum[1] \cite{quelle2}
There has already been done a lot of research when it comes to path planning algorithms. In the following chapter a limited list of implementation is presented in which it is distinguished between past theses, state-of-the-art implementations in robotics and for formula students team around the world. This chapter also introduces the reader to the formula student ZHAW (Zurich university of applied science) association.

\subsection{Formula Student ZHAW Association}
In 1980 the idea of a combustion engine asphalt racing competition was born at the University of Texas by Ron Matthews. Initially it was called the SAE Mini Indy competition but was later renamed to Formula SAE. Since 2010 Formula Student Germany is also part of the SAE.
\cite{formula_sae}

2010 was also the year when the first electric vehicle competition was introduced which was later extended with a self-driving discipline. The self-driving discipline has the same 3 tracks as the normal electric competition with a driver: the acceleration, the skid pad and a track that is unknown to the team. In the competition the team will be rewarded with points in different categories. The team with the highest points wins the whole event. The disciplines are divided by two categories static and dynamic. The static discipline has three subcategories: engineering design (150 points), cost (100 points) and the business plan (75 points). The dynamic discipline is divided into five subcategories: endurance (325 points), efficiency (100 points), autocross (100 points), skid pad (75 points), acceleration (75 points). The point system and how the competition works varies by the competition event. The mentioned points are based on the Formula Student Germany competition. The competition is held around the world and is the biggest student engineering competition. As of today there are approximately 600 teams.
\cite{amz_racing_history}

The formula student ZHAW association was launched in 2019 with 20 members involved. The first fully functional car was build in the year of 2021. Whereas a new car for the season of 2022 is being built. The new car has several improvements in aerodynamics, design and many more. \cite{fszhaw_launch}
To compete against other teams in the driverless category an improved version of the current software in regard to prediction, path planing, localization and many more has to be build. This thesis covers an approach how to implement a path planing or in other word trajectory planing algorithm.
The following topics cover what kind of related work it has been done in this area and state-of-the-art implementation even specifically for the formula student race car autonomous system.

\subsection{Related Work}

Ignat Georgiev a student from the University of Edinburgh did a report on the topic "Path planning and control for an autonomous race car" in his master program. The report was based on the formula student project by the University of Edinburgh. The car follows the see-think-act cycle as an abstract way how the path planning and control process is structured. By "see", he means what information the sensors and cameras provide. With "think", the processing of the images, sensor data, mapping, localization and path planning is meant. And control is the mechanism to steer acceleration and break the vehicle.

For path planning specifically the report mentions two algorithm the A* and the Rapidly-exploring Random Trees (RRT) algorithm. In the DARPA Grand Challenge all the vehicles used the A* Graph Search Algorithm with consideration of the kinematics. Furthermore, the incremental search algorithm "Rapidly-exploring Random Trees (RRT)" was also used in the path in regard to autonomous driving.
For controlling the vehicle another algorithm is used the so-called "Model Predictive Path Integration" (MPPI).
\cite{path_planning_and_control_georgiev}

\subsection{State of the Art}


\subsubsection{Python Implementations}
PythonRobotics is a python library hub for different implementations in robotics. It varies between arm, aerial, car robotics and many more.
In these segments of robotics it provides an overview of different algorithm implementations including path planning algorithms.
The path planning algorithms differs from different approaches for example a dynamic window, breadth first search, Dijkstra just to name a few. 
\cite{python_robotics}

\subsubsection{Formula Student Implementations}




\lipsum[1] \cite{quelle2}

\section{Objective / Problem definition / Requirements}
\begin{itemize}
    \item Formuliert das Ziel der Arbeit
    \item Achtung: Ziel und Aufgabe sind nicht zwingend dasselbe! Bitte sauber trennen.
    \item Verweist auf die offizielle Aufgabenstellung des/der Dozierenden im Anhang
    \item (Pflichtenheft, Spezifikation)
    \item Spezifiziert die Anforderungen an das Resultat der Arbeit
    \item Übersicht über die Arbeit: stellt die folgenden Teile der Arbeit kurz vor
    \item Das erleichtert die Leserführung und schafft Klarheit.
    \item (Angaben zum Zielpublikum: nennt das für die Arbeit vorausgesetzte Wissen)
    \item (Terminologie: Definiert die in der Arbeit verwendeten Begriffe)
    \item Nur spezielle Fachbegriffe; man kann in der Regel von einem informierten Zielpublikum ausgehen.
    Wenn ein Glossar (vgl. 6.2.) erstellt wird, erübrigt sich dieser Abschnitt.
\end{itemize}

\lipsum[1] \cite{quelle2}

\subsection{Subsection}
\lipsum[1] \cite{quelle1}

\begin{figure}[H]
\centering
\includegraphics[width=0.4\columnwidth]{de-zhaw-init-rgb}
\caption{Bildli}
\label{fig:bildli1}
\end{figure}


\subsubsection{SubSubSection}
\lipsum[1]
\begin{table}[H]
\centering
\caption{Eine Tabelle}
\label{tab:my-table}
\begin{tabular}{|l|l|l|}
\hline
\textbf{A} & \textbf{B} & \textbf{C} \\ \hline
1          & 2          & 3          \\ \hline
4          & 5          & 6          \\ \hline
\end{tabular}
\end{table}

\paragraph{Paragraph}
\lipsum[1]

 


