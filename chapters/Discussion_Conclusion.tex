\begin{itemize}
    \item Bespricht die erzielten Ergebnisse bezüglich ihrer Erwartbarkeit, Aussagekraft und Relevanz
    \item Die Diskussion soll von einem differenzierten, sprachlich präzisen Gegenüberstellen von Fakten, Resultaten und Theorien geprägt sein. Persönliche Meinungen haben hier nichts zu suchen! Aussagen müssen durch (mathematische) Logik, wissenschaftliche Theorie oder Statistik begründbar sein. Wenn Vermutungen nicht begründbar sind, so sind diese nur dann festzuhalten, wenn ein Weg zu deren Begründung aufgezeigt werden kann, oder wenigstens eine wissenschaftlich plausible Erklärung existiert.
    \item Interpretation und Validierung der Resultate
    \item Rückblick auf Aufgabenstellung, erreicht bzw. nicht erreicht
    \item Nehmen Sie hier Bezug auf den Abschnitt 1.2!
    \item Legt dar, wie an die Resultate (konkret vom Industriepartner oder weiteren Forschungsarbeiten; allgemein) angeschlossen werden kann; legt dar, welche Chancen die Resultate bieten.
    \item Das weitere Vorgehen ist ebenso wichtig wie Ihre Arbeit. Jede wissenschaftliche Arbeit enthält offene Fragen oder Arbeitsschritte, die aus bestimmten Gründen nicht ausgeführt werden konnten. Diese sind aufzulisten und zu begründen.
\end{itemize}

\lipsum[1]