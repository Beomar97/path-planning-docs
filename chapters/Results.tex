\begin{itemize}
    \item Zusammenfassung der Resultate
    \item Hier geben Sie wieder, was aus der Arbeit als Ergebnis resultiert. Es ist darauf zu achten, dass keine Bewertung der Daten vorweggenommen wird. Diese soll im Diskussionsteil erfolgen. Trotzdem sind die Daten und Resultate mit genügend Text zu erklären. Absolut zentral ist dabei eine präzise, treffende sprachliche Ausdrucksweise. Von Alltagsslang und vagen Ausdrücken ist unbedingt abzusehen.
          Bei grossen Datenmengen müssen die Rohdaten nicht zwingend publiziert werden.
\end{itemize}

This chapter provides the result of the different approaches that have been used during the evaluation phase. Starting off the exploration algorithm tests were made based on visual tests of the Python plots. Then the test results of the optimization algorithm will be presented along with the different output of the program. To round up the integration tests are presented with the test on the simulation tool.

\section{Exploration Algorithm}
\begin{itemize}
    \item verschiedene Screenshot
\end{itemize}
The exploration algorithm test were made via the python plots. CSV files were used from the \acrlong{tum} to have track information like the position of the cones and the car in an abstract data format. \cite{tumftm_optimization_algoritm}

\section{Optimization Algorithm}
\begin{itemize}
    \item verschiedene Screenshot
    \item Zeiten auswertung / vergleich / übersicht
\end{itemize}
Optimization times for berlin\_2018

shortest\_path = 16.72s
Estimated lap time: 89.12s

mincurv = 44.40s
Estimated lap time: 82.46s

mincurv\_iqp = 127.69s
Estimated lap time: 81.06s

mintime = 92.00s
Estimated lap time: 85.77s

\section{Integration}
\begin{itemize}
    \item Test im Python Plots
    \item Simulationstool (Screenshot, evtl. folge von Screenshots), welche strecke (grafik simulationstool)
\end{itemize}

