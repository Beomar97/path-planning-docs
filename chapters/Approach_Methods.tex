\begin{itemize}
    \item Beschreibt die Grundüberlegungen der realisierten Lösung (Konstruktion/Entwurf) und die Realisierung als Simulation, als Prototyp oder als Software-Komponente etc.
    \item Hier beschreiben Sie Ihre gemachte Arbeit. Dazu braucht es eine Beschreibung des Vorgehens, aller Arbeitsschritte usw.
    \item (Definiert Messgrössen, beschreibt Mess- oder Versuchsaufbau, beschreibt und dokumentiert Durchführung der Messungen/Versuche)
    \item Bildmaterial erleichtert das Verständnis.
    \item (Experimente)
    \item Immer mit Aufbau und Vorgehen; Bildmaterial erleichtert das Verständnis.
    \item (Lösungsweg)
    \item Inkl. theoretische Herleitung der Lösung
    \item (Modell)
    \item (Eingesetzte Software)
    \item Die Funktionen von verwendeten Computerprogrammen zu Simulationszwecken, Berechnungen etc. sollen beschrieben werden. Dies soll aber in Worten, Formeln und geeigneten Darstellungen (z.B. Fluss- diagrammen) geschehen. Allfälliger Programmcode ist in einem Anhang zu dokumentieren.
    \item (Tests und Validierung)
\end{itemize}

This chapter describes which approaches and methods have been used to solve the problem to find algorithms which calculate the optimum path on a racetrack. The V-Model was used to plan the overall project.

\begin{figure}[H]
    \centering
    \includegraphics[width=\columnwidth]{High_Level_Project_Overview.png}
    \caption{The high level project overview gives an illustration which approaches and methods have been used to realize the project.}
    \label{fig:High Level Project Overview}
\end{figure}

\section{Marco's Proposal}
\begin{itemize}
    \item Anfangs High Level Overview Mischung, Kapitel ersichtlich was kommt Flussdiagram (um dieses haben wir dann die Prozessmethoden)
    \item Vorgehensmethode => Kanban und Scrum, maybe V-Model?
    \item Setup technisch, lokale Linux VM, ROS Installation, VS Code als IDE, GitHub Repos, GitHub Actions CI/CD (maybe?), Deployment Architektur
    \item Overview Architektur ROS und Code, Grundüberlegung => Path Planning Package in ROS, Prototyp Architektur zeigen, Explo und Opt Algos, erhaltet Input und sendet Input
    \item Messages (interfaces und fszhaw msgs) (System aussen)
    \item Path Planner Node
    \item Exploration Algorithm
    \item Optimization Service Node
    \item Optimization Algorithm
    \item Verifikation und Validierungen (Code Reviews, Fehlerfälle: Cones gehen verloren, Cones andere Seite entdeckt, allg Fehlerannahmen)
    \item Testing mit Maps, Cone Publisher und Planned Trajectory Subscriber (mocks), utils wie track plotter, trackconfig
    \item Setup eher im Projektanhang: Weekly meetings, Review every other week, Mitarbeit mit anderen BA Teams in Driverless und gesamt Verein (Working Saturday, Hilfe im Workshop, Ausstellung Conecto ZHAW)
    \item In Resultate Kapitel, Vergleich erste Algorithmen und Versuche: erste Überlegungen und Tests, alter path planner zhaw, dann densify und interpolate, rrt max hamburg, komplexer algo à la ultimate und dann jetzige implementation, zuerst was hat eth mit mpcc, dann global racetrajectory von tumftm
\end{itemize}

\section{Planning Methods} \label{sec:Planning Methods}
To accomplish complex tasks in a team or alone there has to be a certain plan. In modern Software Engineering the method Scrum is very common as an agile planning method. As for an overall plan the V-Model was used.

\subsection{V-Model} \label{sec:Planning Method: V-Model}
The V-Model consists of several stages as shown in figure \ref{fig:V-Model}. 
\begin{figure}[H]
    \centering
    \includegraphics[width=\columnwidth]{V-Model.png}
    \caption{The V-Model helps in an IT-Project to have a focus on planning and implementation while reflecting and testing the program.}
    \label{fig:V-Model}
\end{figure}

Starting with the concept of operation stage. In this stage the plan is made and research is conducted. Second the requirements and architecture is decided. This covers the process of building a development system that is near to the production environment. In this project the production system is the NVIDIA Jetson and the development system the \acrlong{vm}. Splitting up the problem to find an optimum path into two smaller ones like the exploration and optimization algorithm helped to divide the workload and having the four eye principle on each other's code. The last part covers the implementation of the algorithms in the development environment and testing it. At the same time test definition on how to test the algorithms and how to verify a good one has to be done. The integration test and verification is covered in chapter \ref{ch:Results}. The concept of Operation process consists of the evaluation of the algorithms if they fit the criterions. Lastly the operation and maintenance phase deals with the integration of the algorithm into the production system. This is as well the combination of the exploration and optimization algorithm while using interfaces.

\subsection{Scrum} \label{sec:Planning Method: Scrum}
To ensure the communication between supervisors and developers Scrum was used. Weekly meetings where held with supervisors and biweekly meetings with the driverless team. In addition, the team itself held a meeting on a weekly basis as well. From the meeting notes user stories were created to translate the stories into code. Before the weekly meeting with the supervisors an e-mail is sent which covered the tasks that have been done in the past week, the task for the week after and problems the team faced during development.

\subsubsection{Kanban Board} \label{sec:Kanban Board}
The kanban board helps to organize the user stories and to prioritize the stories in the development process. In addition, another kanban board was made to write the thesis. As shown in figure \ref{fig:Kanban Board Path Planning} there are four columns. The ``To Do'' list with all the ideas and inputs from supervisors. The ``In progress'' list shows the current tasks and the ``Review in progress'' is for the other team member to review the code. Finally, the ``Done'' column illustrates the finished tasks.
\begin{figure}[H]
    \centering
    \includegraphics[width=\columnwidth]{Kanban_Board_Path_Planning.PNG}
    \caption{The Kanban Board helps to organize the user stories to see who is working on which story.}
    \label{fig:Kanban Board Path Planning}
\end{figure}


\section{Architecture Design} \label{sec:Architecture Design}
\begin{itemize}
    \item Beschreibt Architektur von ROS
    \item UML Klassendiagram
    \item Interfaces für Messages
    \item Kommunikationsdiagramm (zwischen Algorithmen)
\end{itemize}

\section{Exploration Algorithm} \label{sec:Exploration Algorithm}
To find the middle line an algorithm has to be used. The exploration algorithm finds the line on which the car should drive on the first round on the track. Exploration is possible with the help of the current position of the car and the cones which are recognized by the Lidar sensor and stereo camera while the car is moving along the track. The \acrshort{ros} Node ``/path\_planner'' subscribes on the ``cone'' and ``current\_position'' topic where it gets all the information needed to calculate the path. For the subscription of ``/current\_position'' a callback function named ``self.\_\_current\_position\_listener\_callback'' will save the position in a local variable. The ``cone'' subscription listens to new cones that will be published via the ``cone\_publisher'' Node.

\section{Optimization Algorithm} \label{sec:Optimization Algorithm}
After the explore algorithm analysed the middle line of the track the board computer switches to the optimization algorithm. For the base code of the algorithm we forked the code from \acrlong{tum} (\acrshort{tum}). \cite{tumftm_optimization_algoritm} 



\section{Integration and Verification}

\begin{itemize}
    \item Tests
    \item Integration (inkl. Switch)
    \item Verifikation über Simulationstool (wie ist das setup aufgebaut, welche strecken wurden getestet)
\end{itemize}