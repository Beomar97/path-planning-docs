\begin{itemize}
    \item Zusammenfassung der Resultate
    \item Hier geben Sie wieder, was aus der Arbeit als Ergebnis resultiert. Es ist darauf zu achten, dass keine Bewertung der Daten vorweggenommen wird. Diese soll im Diskussionsteil erfolgen. Trotzdem sind die Daten und Resultate mit genügend Text zu erklären. Absolut zentral ist dabei eine präzise, treffende sprachliche Ausdrucksweise. Von Alltagsslang und vagen Ausdrücken ist unbedingt abzusehen.
    Bei grossen Datenmengen müssen die Rohdaten nicht zwingend publiziert werden.
\end{itemize}

\lipsum[1]