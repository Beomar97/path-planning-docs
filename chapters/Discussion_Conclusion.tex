\begin{itemize}
    \item Bespricht die erzielten Ergebnisse bezüglich ihrer Erwartbarkeit, Aussagekraft und Relevanz
    \item Die Diskussion soll von einem differenzierten, sprachlich präzisen Gegenüberstellen von Fakten, Resultaten und Theorien geprägt sein. Persönliche Meinungen haben hier nichts zu suchen! Aussagen müssen durch (mathematische) Logik, wissenschaftliche Theorie oder Statistik begründbar sein. Wenn Vermutungen nicht begründbar sind, so sind diese nur dann festzuhalten, wenn ein Weg zu deren Begründung aufgezeigt werden kann, oder wenigstens eine wissenschaftlich plausible Erklärung existiert.
    \item Interpretation und Validierung der Resultate
    \item Rückblick auf Aufgabenstellung, erreicht bzw. nicht erreicht
    \item Nehmen Sie hier Bezug auf den Abschnitt 1.2!
    \item Legt dar, wie an die Resultate (konkret vom Industriepartner oder weiteren Forschungsarbeiten; allgemein) angeschlossen werden kann; legt dar, welche Chancen die Resultate bieten.
    \item Das weitere Vorgehen ist ebenso wichtig wie Ihre Arbeit. Jede wissenschaftliche Arbeit enthält offene Fragen oder Arbeitsschritte, die aus bestimmten Gründen nicht ausgeführt werden konnten. Diese sind aufzulisten und zu begründen.
\end{itemize}
This chapter describes the interpretation of the result from the visual testing on the exploration and optimization algorithm. Firstly the exploration algorithm side by side comparison conclusion is made. Followed up with the different modes on the optimization algorithm and the time comparisons of the calculations.

\section{Exploration Algorithm}
The section of the discussion and conclusion of the algorithm is described in this section. To start the discussion and conclusion of the exploration algorithm the result of the acceleration track is interpreted. After that several tracks are compared or mention to explain the result of the comparison or versatility of the algorithm. To round up the ``Skidpad'' track is interpreted with the result of the applied final implementation of the exploration algorithm.

Starting with the interpretation of the acceleration track showed in the section \ref{sec:Results Acceleration Track}. At the first look every algorithm version can be used to calculate the middle line of the acceleration track. The advantage of the second implementation compared to the first one is in the additional points and smoothening path that draws a line from the beginning to the end which is more precise and gives the car an improved path. The final implementation of the algorithm takes the approach of the second one further and creates additional points from the smoothened path to be even more precise so that the car can drive more accurately. Since the track is the most simplistic one of all that where mentioned there is no more improvement that could be made.

The second track was the ``Small Track'' which is illustrated in section \ref{sec:Results Small Track}. The track has curvature which makes it more complicated for the algorithm to handle it. As seen in the comparison on the image of the first, second and latest implementation of the algorithm the curve on the bottom left of the track shows a huge difference between the versions of the algorithm. The curvature for the first and second approach is too near to the yellow cones which is not feasible in a real environment. An additional consideration of the difference of the calculated path and the real path has to be made as well. If this difference meaning the delta is too big then the car would touch the cones. To conclude only the last implementation is usable for a real environment since it handles the curves in an improved way.

The section \ref{sec:Results Rand Track} and \ref{sec:Results Competition 2021 Track} provides an illustration of a possible track for the third discipline in the competition which is the unknown track. The track of the 2021 competition provides a real example how the exploration algorithm can be applied to get the middle line in the first lap. For both tracks the final implementation of the exploration algorithm is shown and provides evidence that it is feasible to test on the real race car.

The last track shows the final implementation of the exploration algorithm on the ``Skidpad'' track which is described in section \ref{sec:Results Skidpad Track}. As shown in the illustration the algorithm can be used for the ``Skidpad'' track, but further improvement for the curvature has to be made so that the algorithms does two rounds on the right circle and two rounds on the left circle before crossing the finish line. Since every measurement of the track is given in the rules book a static implementation of the middle line could be made as well where the car would follow the manually created track that is saved in a CSV file for example.

To conclude the final implementation of the exploration algorithm can be used for various tracks and can handle narrower curvature and wider ones. Further testing have to be made on the real car in combination with real coordinates of the cones and localization of the car. A further improvement for predicting the route could make the calculation of the middle line faster so that the car is less dependent on the speed meaning that it could overflow the ``path\_planner'' node with published cones. In the testing phase different cone publisher rates were used which did not influence the algorithm where it did not happen that it could not calculate the middle line in the right time.
