Formula Student is a worldwide engineering competition held in different locations across the globe. The goal of every participating team is to develop a race car from scratch to compete against other teams on race day. There are combustion engine and electric motor powered vehicles which will compete in individual categories. The \acrlong{zur} team from the \acrlong{zhaw} built an electric-powered race car with driverless capabilities. Challenges include events involving an actual racing driver and without a driver, where the car has to drive itself autonomously. In the driverless competition, the vehicle needs to be able to finish three different racing tracks without human interference. The competitors know two of the three tracks beforehand, while the last track will be unknown until race day. With cones detected on the track, which mark the track's limits, and the car's position, a path planning algorithm needs to manoeuvre the inside the track. This thesis solves this problem with two different algorithms. The first algorithm, the so-called ``Exploration Algorithm'', calculates the middle line of the track and delivers the data to the second algorithm, the so-called ``Optimization Algorithm'', which optimizes the middle line with the influence of the car's attributes to an optimal racing line. The Exploration Algorithm works on different tracks like the ``Acceleration'' track (a straight line) and Track Drive tracks that are unknown to the team until race day. Further improvements can be made on the ``Skidpad'' track, which consists of two pairs of concentric circles in the shape of an eight, on which the car has to drive two times on the right circle and two times on the left circle. A static implementation, where the coordinates of the track are saved manually, can be used to solve the issues on that particular track.
