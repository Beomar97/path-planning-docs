Formula Student ist ein weltweiter Ingenieur Wettbewerb, welcher an verschiedenen Standorten auf der ganzen Erdkugel von statten geht. Das Ziel von jedem Team ist es, ein Rennauto zu entwickeln, dies gegen andere Teams am Renntag antreten kann. Es gibt verbrennungs- und elektromotor angetriebene Rennautos mit verschiedenen Subkategorien von Wettbewerben. Das \acrlong{zur} Team der Zürcher Hochschule für angewannte Wissenschaften hat einen elektrogetriebenes Rennauto entwickelt. Zwei verschiedene Diszipline gibt es in der Elektroauto Kategorie, die Disziplin mit und welche ohne Rennfahrer, diese ``Driverless'' Kategorie gennant wird. Zwei von drei Strecken, welche abgefahren werden müssen, sind schon gegeben mit ihren Dimensionen und der Startposition des Rennautos. Mit den erfassten Pilonen am Rand der Strecke und der Position des Rennautos kommt ein Pfadplanalgorithmus zum zuge, welcher dem Rennauto erlaubt zwischen den Pilonen zu manövrieren. Diese wissenschaftliche Arbeit gibt zwei Beispiele von Algorithmen, welche in Kombination dieses Problem beseitigen können. Der erste Algorithmus ist der Erforschunsalgorithmus, welcher die Mittellinie der Strecke ausrechnet und diese dem zweiten Algorithmus übergibt Der zweite Algotihmus optimiert die Mittellinie. Dies geschiet zusätzlich mit der Beeinflussung der Autoeckdaten, wie beispielsweise Grösse, Gewicht etc. Der Erkundungsalgorithmus kann für die sogennante ``Acceleration'' Stecke (welche eine gerade Strecke ist) und für unbekannte Strecken angewendet werden. Erweiterte Anpassung des Erlimdimgsalgorithmus für die ``Skidpad'' Strecke muss gemacht werden. Die ``Skidpad'' Strecke ist eine Strecke, welche eine liegende Acht representiert, bei der das Rennauto zweimal auf dem rechten Kreis und zweimal auf dem linken Kreis fahren soll, bis es anschliessend ins Ziel manovriert werden muss. Dieses Problem könnte aber auch mit einer staatischen Implementieren, wie zum Beispiel die vorausgespeicherten Kooordinaten der Mittelline, umgesetzt werden.