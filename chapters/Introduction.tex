\section{Initial situation}

\begin{itemize}
    \item Nennt bestehende Arbeiten/Literatur zum Thema
    \item Stand der Technik: Bisherige Lösungen des Problems und deren Grenzen
    \item «Stand der Technik» ist ein Fachbegriff, der den aktuellen Stand des Wissens im Thema meint. Sie beweisen damit, dass Sie das Fachgebiet kennen und das wesentliche Vorwissen aufgearbeitet haben.
    \item (Nennt kurz den Industriepartner und/oder weitere Kooperationspartner und dessen/deren Interesse am Thema Fragestellung)
\end{itemize}

A lot of research has already been done when it comes to path planning algorithms. In the following chapter, a limited list of implementations is presented, with these algorithms being distinguished between past theses, state-of-the-art implementations in robotics and in Formula Student teams all around the world. This chapter also introduces the reader to the \acrlong{fszhaw} (\acrshort{fszhaw}) association of the \acrlong{zhaw}.

\subsection{Formula SAE Competitions}

In 1980, the idea of a combustion engine asphalt racing competition was born at the University of Texas by Ron Matthews. Initially it was called the SAE Mini Indy Competition, but was later renamed to \Gls{formula_sae}. Since 2010, Formula Student Germany is also a part of the \Gls{sae_international} association.
\cite{formula_sae}

These competitions challenge teams of university students to conceive, design, fabricate, develop, and compete with small, formula-style vehicles. They are held around the world and are one of the biggest student engineering competitions. As of today, there are approximately 600 teams. \cite{sae_student_events}
Each competition is split into two classes: an \acrlong{cv} (\acrshort{cv}) class and an \acrlong{ev} (\acrshort{ev}) class, which was first introduced in 2010. Additionally, vehicles of both classes can participate in an \acrlong{dc} (\acrshort{dc}). After a series of technical inspections, vehicles will compete in a series of static and dynamic events, and the team with the most overall points will win the competition for its class or the \acrshort{dc} respectively. The cars will be driven on three different tracks: an acceleration track, a skid pad track and a track that is not known to the team. The static discipline has three subcategories: engineering design (150 points), cost (100 points) and business plan (75 points). The dynamic discipline is divided into five subcategories: endurance (325 points), efficiency (100 points), autocross (100 points), skid pad (75 points) and acceleration (75 points). The point system, and how the competition works varies by the competition event itself. The mentioned points are based on the Formula Student Germany competition.
\cite{amz_racing_history}

\subsection{Formula Student ZHAW Association}

The \acrlong{fszhaw} association was launched back in 2019 with initially 20 members involved. The first fully functional car was build in the year of 2021. A new car  is being built for the 2022 season. The new car has seen several improvements in aerodynamics, design and more. \cite{fszhaw_launch}
For the first time in the association's history, the team will also compete in the \acrshort{dc} category.
To be able to do that, a new and improved version of the current software, with regard to prediction, path planing, localization and more, has to be build. This thesis covers an approach on how to implement a path planing algorithm, or in other words a trajectory planing algorithm.
In the following sections, related work and state-of-the-art implementations in this area will be covered, even specifically for the Formula Student race car autonomous system.

\subsection{Related Work}

Ignat Georgiev, a student at the University of Edinburgh, did a report on the topic "Path planning and control for an autonomous race car" in his master program. The report was based on the Formula Student project by the University of Edinburgh. The car follows the see-think-act cycle as an abstract way on how the path planning and control process is structured. By "see", he means what kind of information the sensors and cameras provide. With "think", the processing of the images, sensor data, mapping, localization and path planning is meant. And as for "act", the mechanism to steer, accelerate and break the vehicle is meant.

For path planning specifically, the report mentions two algorithms: the A* algorithm and the \acrlong{rrt} (\acrshort{rrt}) algorithm. In the DARPA Grand Challenge, all vehicles used a variety of the A* Graph Search Algorithm, with consideration of the kinematics. Furthermore, the incremental search algorithm \acrshort{rrt} was also used in the path in regard to autonomous driving.
To control the vehicle, another algorithm is used, the so-called \acrlong{mppi} (\acrshort{mppi}).
\cite{path_planning_and_control_georgiev}

\subsection{State of the Art}
This section gives a brief overview of different implementation of path planning algorithms. We differ between raw Python implementations and implementations by other Formula Student teams.

\subsubsection{Python Implementations}
PythonRobotics is a Python library hub for a wide range of different implementations in robotics. It varies between arm, aerial, car robotics and more.
For these segments of robotics, it provides an overview of different algorithm implementations, including several path planning algorithms.
The path planning algorithms differ from different approaches, dynamic window, breadth first search, Dijkstra algorithm, just to name a few. 
\cite{python_robotics}

\subsubsection{Formula Student Implementations}
The first student in the \acrshort{fszhaw} team to work on path planning was Jérôme Perdrizat. The implemented path planner consists of a middle line detection, path smoothening and a velocity / acceleration Profile. The work was done as part of his master thesis. To evaluate the middle line of the track, Delaunay-Triangulation was used. An error rate was also mentioned in the thesis, e.g. when a cone wasn't detected. The thesis does not cover an algorithm which draws the optimum line in consideration of curves and speed.
\cite{autopilot_for_formula_student_jerome}

The \acrshort{amz} (\acrlong{amz}) from \acrshort{eth} (\acrlong{eth}) made Formula Student driverless resources available in a public repository, which also includes implementations of path planning algorithms. Several algorithms are being mentioned, e.g. the \acrshort{rrt} algorithm for track exploration and a \acrlong{mpcc} (\acrshort{mpcc}) for autonomous racing. Furthermore, for easier software distribution, a skeleton for building and implementing driverless algorithms is provided.
\cite{amz_racing_github}

\section{Objective / Problem definition / Requirements}
\begin{itemize}
    \item Formuliert das Ziel der Arbeit
    \item Achtung: Ziel und Aufgabe sind nicht zwingend dasselbe! Bitte sauber trennen.
    \item Verweist auf die offizielle Aufgabenstellung des/der Dozierenden im Anhang
    \item (Pflichtenheft, Spezifikation)
    \item Spezifiziert die Anforderungen an das Resultat der Arbeit
    \item Übersicht über die Arbeit: stellt die folgenden Teile der Arbeit kurz vor
    \item Das erleichtert die Leserführung und schafft Klarheit.
    \item (Angaben zum Zielpublikum: nennt das für die Arbeit vorausgesetzte Wissen)
    \item (Terminologie: Definiert die in der Arbeit verwendeten Begriffe)
    \item Nur spezielle Fachbegriffe; man kann in der Regel von einem informierten Zielpublikum ausgehen.
    Wenn ein Glossar (vgl. 6.2.) erstellt wird, erübrigt sich dieser Abschnitt.
\end{itemize}

\lipsum[1] \cite{quelle2}

\subsection{Subsection}
\lipsum[1] \cite{quelle1}

\begin{figure}[H]
\centering
\includegraphics[width=0.4\columnwidth]{de-zhaw-init-rgb}
\caption{Bildli}
\label{fig:bildli1}
\end{figure}


\subsubsection{SubSubSection}
\lipsum[1]
\begin{table}[H]
\centering
\caption{Eine Tabelle}
\label{tab:my-table}
\begin{tabular}{|l|l|l|}
\hline
\textbf{A} & \textbf{B} & \textbf{C} \\ \hline
1          & 2          & 3          \\ \hline
4          & 5          & 6          \\ \hline
\end{tabular}
\end{table}

\paragraph{Paragraph}
\lipsum[1]

 


