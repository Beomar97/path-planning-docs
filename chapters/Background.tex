\begin{itemize}
    \item In der Regel ist zumindest ein kurzes Theoriekapitel notwendig. Es nimmt Bezug auf das thematische Oberthema, aber natürlich nicht auf allgemeine theoretische Grundlagen etwa aus der Naturwissenschaft.
\end{itemize}
The background chapter describes the theory of the approach and methods that was used in the thesis. In the first sections, an introduction into the Formula SAE Competition and \acrlong{zur} is made. After that, the concept of driverless is introduced, what that exactly means and how something like that can be realized. Later on, several path planning algorithms will be introduced and will be followed by an overview of the development environment for the realization of this work.

\section{Formula SAE Competitions} \label{sec:Formula SAE Competitions}
The main idea of a Formula SAE competition is to conceive, design, fabricate, develop and compete with small, formula style, race cars.
The competition is split into two classes: Internal Combustion Engine Vehicle (\acrshort{cv}) and Electric Vehicle (\acrshort{ev}).
Additionally, a team can opt-in to take part in the Driverless Cup (\acrshort{dc}).
After a series of technical inspections in regards for safety and rules compliance, the vehicles will then compete in a series of static and dynamic events. In the end, the team with the most overall points will win the competition for its class or the \acrlong{dc}, respectively. \cite{fs_rules_2022_handbook}
In a single Formula Student Germany competition dozens of teams are participating as shown in figure \ref{fig:FS SAE Competition}.
\begin{figure}[H]
    \centering
    \includegraphics[width=\columnwidth]{FS_SAE_Competition.jpeg}
    \caption{Teams participating at Formula Student Germany \cite{fs_germany}}
    \label{fig:FS SAE Competition}
\end{figure}

\subsection{Static Events} \label{sec:Static Events}
The following static events are held: Business Plan Presentation, Cost and Manufacturing, and Engineering Design. In these events, the engineers have to present their car and their development processes to a panel of judges. \cite{fs_rules_2022_handbook}

\textbf{Business Plan Presentation:} The team's ability to develop and deliver a comprehensive business model will be evaluated. The presentation should demonstrate how their self-developed race car could become a profitable business idea.

\textbf{Cost and Manufacturing:} The financial planning of the car, including the manufacturing processes and costs associated with the construction of the race car are evaluated.

\textbf{Engineering Design:} Evaluation of the engineering process and effort that went into the design of the vehicle. Technical aspects, the construction and key attributes of the car will be judged.

\subsection{Dynamic Events} \label{sec:Dynamic Events}
The following dynamic events are held: Skid Pad, Acceleration, Autocross, Endurance and Efficiency, and Trackdrive.
The dynamic events reveal the driving performance of the prototypes. Every discipline puts different abilities of the cars to the test. \cite{fs_rules_2022_handbook}

\textbf{Skid Pad:} The skid pad track as shown in figure \ref{fig:FS Skid Pad layout} consists of two pairs of concentric circles in the shape of an eight. This track will test the lateral grip of the car.
\begin{figure}[H]
    \centering
    \includegraphics[width=\columnwidth]{FS_Event_Skid_Pad.png}
    \caption{Skid Pad track layout according to the Formula Student Rules 2022 handbook. \cite{fs_rules_2022_handbook}}
    \label{fig:FS Skid Pad layout}
\end{figure}

\textbf{Acceleration:} An acceleration race over 75 m distance as shown in figure \ref{fig:FS Acceleration layout} with a standing start, which tests the car acceleration in a straight line.
\begin{figure}[H]
    \centering
    \includegraphics[width=\columnwidth]{FS_Event_Acceleration.png}
    \caption{Acceleration track layout according to the Formula Student Rules 2022 handbook. \cite{fs_rules_2022_handbook}.}
    \label{fig:FS Acceleration layout}
\end{figure}

\textbf{Autocross:} The autocross event tests the cars dynamic ability in a one lap sprint. The objective of the autocross event is to evaluate the car's manoeuvrability and handling qualities.

\textbf{Endurance and Efficiency:} An endurance race over a distance of 22 km, including one driver change. One lap of the endurance track is approximately 1 km. The Efficiency scoring rates the consumed amount of energy in relation to the total time.

\textbf{Trackdrive (DC Only):}  Over a distance of 10 rounds, the car has to prove its durability without a driver under long-term conditions. As shown in figure \ref{fig:FS Autocross, Endurance and Trackdrive layout} a basic layout is given as a reference.
\begin{figure}[H]
    \centering
    \includegraphics[width=\columnwidth]{FS_Event_Trackdrive.png}
    \caption{Base layout for the Autocross, Endurance and Trackdrive events according to the Formula Student Rules 2022 handbook. \cite{fs_rules_2022_handbook}}
    \label{fig:FS Autocross, Endurance and Trackdrive layout}
\end{figure}

\section{Zurich UAS Racing Autonomous System} \label{sec:Zurich UAS Racing Autonomous System}
On a high-level view, the 'Autonomous System' is made up of five domains: the 'Controls', 'Perception', 'Localization and Mapping', 'Path planning' as well as 'Simulation'. Figure \ref{fig:AS Component Diagram} shows an abstraction of how the different components interact with each other.
\begin{figure}[H]
    \centering
    \includegraphics[width=\columnwidth]{AS_Component_Diagram.png}
    \caption{The Autonomous System Component Diagram shows an illustration between the hardware layer and software layer. The autonomous system is on the software layer in which the 'Path Planning' domain lies.}
    \label{fig:AS Component Diagram}
\end{figure}
The five domains are described as followed:
\textbf{'Controls'} is responsible for the steering, breaking and acceleration of the car.
\textbf{'Perception'} is recognizing the track by perceiving the different types of cones the track is made up of.
\textbf{'Localization and Mapping'} estimates the current position of the vehicle relative to the starting position and maps it into an absolute position on the track.
\textbf{'Path Planning'} computes the best possible path along the track.
\textbf{'Simulation'} is responsible for providing the rest of the domains an adequate tool to simulate and test on.


Essential hardware components of the autonomous system are stored inside a box, which will be mounted on the vehicle itself. The connections coming into the box from the various components outside can be seen in the reference figure \ref{fig:AS DV Box}.

\begin{figure}[H]
    \centering
    \includegraphics[width=\columnwidth]{AS_DV_Box.png}
    \caption{The Autonomous System DV Box consists of different components and connections to sensors, cameras and several other hardware components.}
    \label{fig:AS DV Box}
\end{figure}

Data needed for 'Localization' is received from the \acrshort{gnss} sensor and antenna (a MIKROE GNSS 7 Click and a u-blox ANN-MB), while the inputs for 'Perception' are received from the stereo camera (a Stereolabs ZED 2) and Lidar sensor (a Velodyne Lidar Puck Hi-Res), which both are also needed for the 'Path Planning' module. Information for the 'Control' unit are sent over to the engine control unit (ECU) (a dSPACE MicroAutoBox III). In the end, all critical data for the operation leads into the processing unit of the system, which is in this case a 'Jetson AGX Xavier' by NVIDIA. The Jetson is an AI computer made for autonomous machines, delivering workstation performance in an embedded module under 30 Watts of power. These components are illustrated on figure \ref{fig:AS HW Components}.
\begin{figure}[H]
    \centering
    \includegraphics[width=\columnwidth]{AS_HW_Components.png}
    \caption{The Autonomous System Hardware Components consists of the Stereolabs ZED 2, MIKROE GNSS 7 Click, u-block AHN-MB, Velodyne Lidar Puck Hi-Res, dSPACE MicroAutoBox III and the NVIDIA Jetson AGX Xavier.}
    \label{fig:AS HW Components}
\end{figure}

The processing unit is set up with an installation of Ubuntu Linux 20.04, which will run the software of the autonomous system on top of \acrshort{ros}. Figure \ref{fig:AS Deployment Diagram} shows the levels of the software layers used.
\begin{figure}[H]
    \centering
    \includegraphics[width=\columnwidth]{AS_Deployment_Diagram.png}
    \caption{The Autonomous System Deployment Diagram illustrates different sensors and how they are connected over a bus with the NVIDIA Jetson computer.}
    \label{fig:AS Deployment Diagram}
\end{figure}

\section{Driverless} \label{sec:Driverless}
\lipsum[1]

\section{Path Planning Algorithms} \label{sec:Path Planning Algorithms}
Path planning algorithms are used in many environments, e.g. in the assembly of a car with a robot arm, in the autonomous parking of a car or in finding a landmine with a robot during a military operation.

\subsection{Overview} \label{sec:Overview Path Planning Algorithms}
A path can be, if a human goes from point A to point B while following certain signalizations, e.g. when hiking on a mountain. Otherwise, a human can follow a path given by them by a GPS system in a car. The difference lies between a manual plan versus a generated plan by a machine. This section addresses the later one.

The planning of a path consists of a state, the time that is needed, actions, an initial / goal state, a criterion and a plan. The state is normally represented by a planning algorithm. Time is based on a sequence of decision that has to be made in a certain time. Actions manipulate the state and cover how the state should be changed. The initial and goal state defines where the planner should start and finish.
The criterion defines the outcome and the boundaries of the planner. The type of the criterion is either based on the feasibility or optimality. Feasibility covers the arrival of the goal state without the concern of efficiency. Optimality finds a plan with an optimized performance in mind. The plan in general covers the strategy or behaviour on a decision maker. \cite{planning_algorithms_steven_m_lavalle}

A path planning algorithm can be categorized in one of the following groups: Motion Planning, Decision-Theoretic Planning and Planning under Differential Constraints.
\cite{planning_algorithms_steven_m_lavalle}

\subsection{Algorithms, Planners and Plans} \label{sec:Algorithms, Planners and Plans}
An algorithm usually consists of a machine and environment where there are actuations from the machine to the environment and sensing from the environment to the machine as shown in figure \ref{fig:Machine and Environment interaction}.
\begin{figure}[H]
    \centering
    \includegraphics[width=\columnwidth]{Machine_Environment.png}
    \caption{Two different illustrations that show what is the relationship between the machine and the environment. \cite{planning_algorithms_steven_m_lavalle}}
    \label{fig:Machine and Environment interaction}
\end{figure}
The problem in planning is often that the machine interacts with the physical world, which has uncountable different influences on the plan that has to be calculated. This is why the environment is often an approximation of the real world, since not every sensing mechanism from the environment to the machine can be processed.

Planners construct plans and are either a machine or a human. Should the planner be a machine, then a planning algorithm would be the planner. Should the planner be a human, then the human itself would be the algorithm making the decisions. Figure \ref{fig:Planner Machine Environment} shows the interaction of a planner on a machine or a plan.
\begin{figure}[H]
    \centering
    \includegraphics[width=5cm]{Planner_Machine_Environment.png}
    \caption{A planner executes the plan on the machine that actuates on the environment. \cite{planning_algorithms_steven_m_lavalle}}
    \label{fig:Planner Machine Environment}
\end{figure}

These plans can be used in three different ways: Execution, refinement or hierarchical inclusion. Execution runs the plan on a simulation or a robot connected to the real world. Refinement tries to find a better plan. Hierarchical inclusion hands over the plan to a higher level plan consisting of additional sub plans. \cite{planning_algorithms_steven_m_lavalle}

\subsection{Motion Planning} \label{sec:Motion Planning}

Motion planning is interested in the planning in continuous state spaces. This means that space is not static and will have changes in relation to time. It is also referred to as motion planning or planning in continuous state spaces. \cite{planning_algorithms_steven_m_lavalle}

\textbf{Implicit Representation} of a state space in motion planning must be dealt with in planning algorithms. Implicit representations become more important in motion planning as the state space is uncountably infinite. It also tries to define 2D and 3D geometric models and transforming them. The state spaces arise from these problems. \cite{planning_algorithms_steven_m_lavalle}

\textbf{Continuous $\rightarrow$ discrete} is the central theme in motion planning to transform models. Motion planning also covers combinatorial motion planning, where the algorithm builds a representation of the original problem with an input model. Additionally, there exists also the field of sampling-based motion planning. \cite{planning_algorithms_steven_m_lavalle}

\textbf{Geometric Representation and Transformation} to start defining a motion planning algorithm. It defines a body of a system in a space and how it transforms, e.g. moves. Movements can be chained in a state. \cite{planning_algorithms_steven_m_lavalle}

\textbf{The Configuration Space} is needed for the motion planning algorithm to define the space, in which a set of possible transformations could be applied to the robot. \cite{planning_algorithms_steven_m_lavalle}

\textbf{Sampling-Based Motion Planning} is a field in motion planning that covers sampling-based algorithms. Sampling-based planning algorithm use collision detection as a "black box" separated from the geometric and kinematic models of motion planning. There are two different types of algorithm models. First, there is the single-query algorithm, which consists of a ($q_i,q_G$) pair as an input. $q_i$ defines the start and $q_G$ the end goal. In this strategy, no pre-computation is possible to make use of. Figure \ref{fig:Geometric Models Collision Detection Algorithm} illustrates the interaction between the geometric models, collision detection, the sampling-based motion planning algorithm and the discrete searching method.
\begin{figure}[H]
    \centering
    \includegraphics[width=\columnwidth]{Sampling-Based_Motion_Planning.png}
    \caption{The geometric models will generate the input for the collision detection, that influences the discrete searching in the sampling-based motion planning algorithm. The C-space is the configuration space of the algorithm. \cite{planning_algorithms_steven_m_lavalle}}
    \label{fig:Geometric Models Collision Detection Algorithm}
\end{figure}
A multi-query algorithm can have multiple initial goals, that's when it makes sense to precompute the models for more efficiency. An example of a single-query algorithm is the \acrlong{rdt} (\acrshort{rdt}), which is a subset of the \acrlong{rdt} (\acrshort{rdt}) algorithm. It searches for the shortest path by creating random sequences that end up in a tree, which itself holds multiple sequences which can be connected to each other. Figure \ref{fig:Rapidly Exploring Dense Tree} illustrates a simple example how a \acrshort{rdt} works.
\begin{figure}[H]
    \centering
    \includegraphics[width=\columnwidth]{Rapidly_Exploring_Dense_Tree.png}
    \caption{$q_0$ is the initial position and $q_n$ the vertex. $\alpha(i)$ is the sample based on the nearest point in $S$. (a) shows the tree that has been constructed and (b) the tree after applying the algorithm. \cite{planning_algorithms_steven_m_lavalle}}
    \label{fig:Rapidly Exploring Dense Tree}
\end{figure}

Listing \ref{lst:Simple RDT} shows a pseudocode implementation of the \acrshort{rdt} algorithm.
\begin{lstlisting}[mathescape=true, caption={The simple \acrshort{rdt} computes a random tree with the nearest function. \cite{planning_algorithms_steven_m_lavalle}}, label={lst:Simple RDT}]
SIMPLE RDT($q_0$)
 1 G.init($q_0$);
 2 for i = 1 to k do
 3 G.add vertex($\alpha(i)$);
 4 $q_n \rightarrow$ nearest(S(G), $\alpha(i)$);
 5 G.add edge($q_n$, $\alpha(i)$);
\end{lstlisting}
The goal of the multi-query  method is to create a roadmap for each $q_i$ and $q_G$, the reason why the family of algorithm is called sampling-based roadmap. \cite{planning_algorithms_steven_m_lavalle}

\textbf{Combinatorial Motion Planning} covers the discovery of a path in a continuous space without resorting to approximations. Cell decompositions and cylindrical algebraic decomposition are two different subcategories of combinatorial motion planning algorithms. Cell decompositions are based on a collection of cells called a complex. In the 2D decomposition field exists a triangulation algorithm performed by vertical decomposition. The algorithm connects every built triangle with three nodes at a time. With four nodes, two triangles are made. Figure \ref{fig:Theory of Triangulation} gives an example on how the edges from the result of the triangulation are connected through the middle points.
\begin{figure}[H]
    \centering
    \includegraphics[width=10cm]{Theory_of_Triangulation.png}
    \caption{The black lines represent the triangles which are computed. The red dots on the other hand represent the middle point of the edges and the triangle itself. The red lines have the red dots as a reference point and can build a path. \cite{planning_algorithms_steven_m_lavalle}}
    \label{fig:Theory of Triangulation}
\end{figure}
In computational algebraic geometry, a very general definition is given and is the reason why it can cover a lot of solutions for problems, but at the same time are challenging to implement. The definitions are described in tarski sentences. The problems that can exist with this method are the  decision problem and the quantifier-elimination problem. The explanation of each problem is out of scope. The Canny's roadmap algorithm, a subset of computational algebraic geometry algorithm, covers the avoidance of doubly exponentially cells in cylindrical algebraic. \cite{constructing_roadmaps_of_semi-algebraic_sets_I} The result is an algorithm that can be run in polynomial time. The complexity of motion planning describes if it makes sense for an algorithm to be programmed more complexed or to leave it as it is timewise or storage wise. Optimal motion planning is different to feasibility, as it covers to find an optimal solution of the problem. It deals with finding an approximation of a continuous problem as a discrete problem. \cite{planning_algorithms_steven_m_lavalle}

\textbf{Extensions of Basic Motion Planning} defines flavours of motion planning algorithms. There are for example the time-varying problems, which are defined in a planning formulation. Time-varying problems consider time in the planning formulation. For example, obstacles can change during time and the space would not be considered as static. Solutions for that are either sampling-based, combinatorial methods or bounded speed. Furthermore, there can be multiple robots in a space. With multiple robots, it is possible to combine the information of each robot, so that one can benefit from the other. Otherwise, there is also the problem that the robots do not collide with each other. \cite{planning_algorithms_steven_m_lavalle} Euclidean shortest path algorithm gives a solution for solving a motion plan. \cite{efficient_computation_of_euclidean_shortest_paths_in_the_plane}

\textbf{Feedback Motion Planning} describes a class of motion planning algorithms that are taking feedback, for example from the real world, into account. With the help of the Dijkstra algorithm, an optimal navigation function can be programmed to work in a feedback motion planning environment, with acceleration in consideration for example. \cite{planning_algorithms_steven_m_lavalle}

\subsection{Decision-Theoretic Planning} \label{sec:Decision-Theoretic Planning}

Decision-Theoretic Planning can also be called planning under uncertainty. Two aspects evolve under uncertainty: predictability and sensing. The interference of the algorithm with another decision maker could be possible under these circumstances. Instead of decision-making plans, the topic covers strategies for a decision maker. While nature can be a decision maker as well, and it is already possible to predict certain things about nature, the factors of interference of nature is still too big. Therefore, the programmer has to concentrate on so-called primary decision-makers. \cite{planning_algorithms_steven_m_lavalle}

\textbf{Sequential Decision Theory} deals with forward and backward projections. A forward projection predicts how the future route will look like. Forward projection is a method where several possible stages are computed before it decides which path to go. Backward projection is the other way around. With the help of graph search it, is possible to have a more stable solution for calculating a path then with other methods like value iteration or policy iteration. Infinite-horizon problems can occur when dealing with algorithms under the sequential decision theory. Reinforcement Learning helps to solve infinite-horizon problems in combination with the optimal path in one algorithm. \cite{planning_algorithms_steven_m_lavalle}

\textbf{Sensors and Information Spaces} describes how the sensors act and the planning problem witch is described as an information space. There are discrete state spaces and derived information spaces. Sensors have two components, the observation space that defines possible states and a sensor mapping which is what the sensor reads. \cite{planning_algorithms_steven_m_lavalle}

\textbf{Planning Under Sensing Uncertainty} covers algorithms which can be used to plan under sensing uncertainty. Localization as a problem in robotics can be hard to accomplish. There is the passive localization and the active localization. In passive localization, a robot acts upon different circumstances and derives its position from probabilistic approaches, whereas in active localization the focus is more on how the robot should move to be in that exact position. Figure \ref{fig:Localization Example} gives an example of a robot that interacts on a grid map and shows the possible moves he can do. \cite{planning_algorithms_steven_m_lavalle}
\begin{figure}[H]
    \centering
    \includegraphics[width=3cm]{Localization_Example.png}
    \caption{This example shows possible moves of a robot to which grids he could move to. \cite{planning_algorithms_steven_m_lavalle}}
    \label{fig:Localization Example}
\end{figure}

With the combination of different sensors like a compass and a camera, it is possible to define a position. Further information on localization can be read in the mentioned book. The environment uncertainty and mapping topic focuses on how to define an environment when there is no map given. This means that the robot has to act upon sensors only. \cite{planning_algorithms_steven_m_lavalle}

\subsection{Planning under Differential Constraints} \label{sec:Planning under Differential Constraints}
This chapter covers differential constraints. These for example local limitation like allowable velocities at each point. A weak differential constraint is for example smoothness. Global constraints are for example obstacles. Mostly differential constraints are from kinematics and dynamics.

\textbf{Differential Models} covers example of velocity constraints in a state space. There are two ways to of differential constraints the parametric and implicit. Implicit expresses prohibited velocities and are more general whereas parametric representation expresses allowed velocities. A simple car has three degrees of freedom as shown in figure \ref{fig:Simple Car Velocity Space}.
\begin{figure}[H]
    \centering
    \includegraphics[width=6cm]{Simple_Car_Velocity_Space.png}
    \caption{This example shows possible moves of a robot to which grids he could move to. \cite{planning_algorithms_steven_m_lavalle}}
    \label{fig:Simple Car Velocity Space}
\end{figure}
The velocity is two-dimensional on the other hand. This means that it has to be shrunken down from three dimensions to two. There are also concepts of multiple decision makers, which are not covered in this thesis.

\textbf{Sampling-Based Planning Under Differential Constraints} covers the satisfaction of differential constraints with an initial and goal state. Constraints like continuity or smoothness can be applied as well. Kinodynamic planning is a motion planning problem with velocity and accelerations bounds. Trajectory planning on the other hand deals with a velocity and path function. There can also be collision constraints with obstacles for example.
Reachable sets describe what has to be visited, there could also be time-limited reachable sets. A reachability graph can help to visualize which sets are already visited like the one in figure \ref{fig:Reachability Graph}.
\begin{figure}[H]
    \centering
    \includegraphics[width=4cm]{Reachability_Graph.png}
    \caption{This figure shows how a reachability graph can be visualized and represents two stages. \cite{planning_algorithms_steven_m_lavalle}}
    \label{fig:Reachability Graph}
\end{figure}
Local planning by the having smaller problems which can be solved locally to solve the bigger problem. The following pseudocode snipped in listing \ref{lst:Simple RDT with differential constraints} shows how the \acrlong{rdt} (\acrshort{rdt}) can be used for calculating a path under differential constraints.
\begin{lstlisting}[mathescape=true, caption={The local planning method computes $x_r$. A new vertex will be available: $x_r$. \cite{planning_algorithms_steven_m_lavalle}}, label={lst:Simple RDT with differential constraints}]
SIMPLE RDT WITH DIFFERENTIAL CONSTRAINTS($x_0$)
 1 G.init($x_0$);
 2 for i = 1 to k do
 3 $x_n$ $\leftarrow$ nearest($S(G), \alpha(i)$);
 4 ($\widetilde{u}^p$, $x_r$) $\leftarrow$ local planner($x_n$, $\alpha(i)$);
 5 G.add vertex($x_r$);
 6 G.add edge($\widetilde{u}^p$);
\end{lstlisting}
Randomized potential fields can also be a method for sampling-based planning under differential constraints, which is not covered in this thesis. Path constraints can narrow down the trajectory as well, like the turning angle. \cite{planning_algorithms_steven_m_lavalle}

\textbf{System Theory and Analytical Techniques} takes the physics of the robot into account. Since this thesis covers trajectory planning or path planning this section would be out of scope. \cite{planning_algorithms_steven_m_lavalle}

\section{Development Environment} \label{sec:Development Environment}

The development environment is made up from different software to help the workflow of developing path planning algorithms for a robot like a self-driving racecar.

\subsection{Hardware Abstraction} \label{sec:Hardware Abstraction}

There are different methods to build a development environment. An easy way to get started is to build a relatively similar system like the one which is running on the NVIDIA Jetson computer. This computer will be used in the real car. A \acrlong{vm} (\acrshort{vm}) is a piece of software that can emulate hardware and encapsulates an operating system from the host operating system, that is running on hardware. Figure \ref{fig:Development Architecture} illustrates the architecture that is used for developing.

\begin{figure}[H]
    \centering
    \includegraphics[width=2cm]{Development_Architecture.jpg}
    \caption{The development architecture consists of various levels: the hardware, the host operating system, the \acrlong{vm}, the operating system on the \acrshort{vm} and the \acrlong{ros}.}
    \label{fig:Development Architecture}
\end{figure}

On the NVIDIA Jetson computer an Ubuntu 20.04 operating system was installed. This is the reason why Ubuntu 20.04 is used on the development environment. An installation guide for how to install Ubuntu 20.04 in a \acrlong{vm} is found Cloud Linux Tech. \cite{cloudlinuxtech_install_ubuntu_2004}

\begin{itemize}
    \item ROS 2
    \item Python
    \item LaTeX
    \item Git
    \item VS Code
    \item Azure DevOps
    \item GitHub
    \item Simulation tools
    \item other tools
\end{itemize}

\subsection{Robot Operating System (ROS)} \label{sec:Robot Operating System (ROS)}
The \acrlong{ros} (\acrshort{ros}) is not, like the name may suggest, a full-fledged operating system, but a set of software libraries and tools for the development of robot applications. The open-source robotics middleware comes shipped with capable developer tools, drivers and advanced algorithms. \cite{ros2_documentation}

There are currently two major versions of \acrshort{ros} which are seeing releases, ROS 1 and ROS 2. \cite{ros2_documentation} Beginning with releases after 'Foxy Fitzroy', releases in odd years will be non-LTS (Long Term Support) and will only be supported for 1.5 years, while new releases in even years are going to be long-term supported and will be supported for 5 years. \cite{ros2_documentation}

The work done in this thesis have been done using the ROS 2 release 'Foxy Fitzroy', released on June 5th, 2020. This release will be supported till the end of May 2023. \cite{ros2_releases_and_target_platforms}

\subsubsection{ROS Graph} \label{sec:ROS Graph}
There are 5 main concepts of ROS 2 that make up the ROS (2) graph:
\begin{enumerate}
    \item Nodes
    \item Topics
    \item Services
    \item Parameters
    \item Actions
\end{enumerate}

The ROS graph is a network of ROS 2 elements which processes data simultaneously. The graph encompasses all executables and the connections between them.

\begin{figure}[H]
    \centering
    \includegraphics[width=\columnwidth]{ROS2_Main_Concepts.png}
    \caption{The five main concepts of ROS 2 pictured as a network of nodes.}
    \label{fig:ROS 2 main concepts}
\end{figure}

\subsubsection{Nodes} \label{sec:ROS Nodes}
A node is a fundamental ROS 2 element that serves a single, modular purpose in a robotics system. \cite{ros2_documentation}

\subsubsection{Topics} \label{sec:ROS Topics}
Nodes publish information over topics, which allows any number of other nodes to subscribe to and access that information. \cite{ros2_documentation}

\subsubsection{Services} \label{sec:ROS Services}
Services are based on a call-and-response model, versus topics’ publisher-subscriber model. Services only provide data when they are specifically called by a client. \cite{ros2_documentation}

\subsubsection{Parameters} \label{sec:ROS Parameters}
Nodes have parameters to define their default configuration values. \cite{ros2_documentation}

\subsubsection{Actions} \label{sec:ROS Actions}
Actions are built on topics and services and consist of a goal, feedback, and a result. Actions are like services that allow you to execute long-running tasks, provide regular feedback, and are cancellable. \cite{ros2_documentation}

\section{Scrum} \label{sec:Scrum}
Scrum is a framework which can be used by people, teams and organizations to solve problems through adaptive solutions. Four simple steps describe the process of Scrum which is lead by a Scrum Master:
\begin{enumerate}
    \item The Product Owner orders the work for a problem and puts it into a Product Backlog.
    \item The Scrum Team selects certain tasks from the Backlog, gives a number based on a certain expense and tries to realize it in a Sprint.
    \item The Scrum Team and stakeholders analyse the result and adjust it for the next sprint.
    \item After this the above steps will be repeated.
\end{enumerate}

These simplified steps help to explain how Scrum works. Scrum consists of several team roles which made up the Scrum Team. Scrum Events define what has to be done in each type of meeting. The figure \ref{fig:Scrum Framework} shows the Scrum Framework and gives an overview of the Events and Artefacts of Scrum. \cite{scrum_guide}


\begin{figure}[H]
    \centering
    \includegraphics[width=\columnwidth]{Scrumorg_Scrum_Framework.png}
    \caption{The Scrum Framework starts with the Product Backlog and will give through the Sprint Planning certain tasks to the Sprint Backlog. The developers in the Scrum Team work on the tasks in a Sprint and define the Increment of each Sprint. After the Sprint is finished a Sprint Review will be made. \cite{scrum_guide}}
    \label{fig:Scrum Framework}
\end{figure}


\subsection{Scrum Team} \label{sec:Scrum Team}
The Scrum Team consists of one Scrum Master, a Product Owner and Developers. Normally a Scrum Team has 10 or fewer people. There are other methods to plan like Scrum on a corporate or with large amount of peoples. If the team gets to large they should consider splitting the team in smaller ones. Bellow are all the team roles with the task their doing in a Scrum Team. \cite{scrum_guide}

\textbf{Developers} are the people that create the product. Several skills are needed to work as a developer. For the Scrum specific skills they have to create a plan for the Sprint which is called the Sprint Backlog. Furthermore, they have to create methods to maintain the quality to be sure what 'Done' means in a Sprint. Next they have to adapt their plan every day until the Sprint Goal is reached. One of the most important tasks for developers is to hold themselves accountable on a professional level. \cite{scrum_guide}

\textbf{Product Owner} is responsible for the Product Backlog. The tasks of a Product Owner consist of explicitly communication the Product Goal, to create and communicate each item in the Product Backlog, orders the Product Backlog items and checks if the Product Backlog is understood by every team member. \cite{scrum_guide} The Product Owner can be a working college or an employer from another firm.

\textbf{Scrum Master} is the role where the person ensures that all members are following the Scrum Guide. The theory and practices has to be understood by every member. Self-management and cross-functionality are the abilities that the Scrum Master coaches on the team. The Scrum Master helps the developers to define 'Done' in each Sprint and ensures that every Scrum Event is held within the time frame. In terms of the Product Owner the Scrum Master helps to define the Product Goal and Backlog and serves as a middle man between the Product Owner and the developers so that everything is understood. Furthermore, he collaborates with the stakeholders if there are any requests or needs. In the Organization a Scrum Master is responsible to get employers trained on Scrum, planning and advise the implementation, makes sure that everyone understand how Scrum works and removes walls between stakeholders and the Scrum Teams. \cite{scrum_guide}

\subsection{Scrum Events} \label{sec:Scrum Events}
Scrum has so-called Scrum Events that are minimizing the need for regular meetings and to focus on transparency. The events are as followed:
\textbf{The Sprint} is the foundation of Scrum. They are normally held once a month or less. The durance of a Sprint durance is as long until another Sprint starts. In a Sprint the Product Goal, Sprint Planning, Daily Scrums, the Sprint Review and the Retrospective will be defined. During the Sprint it is not allowed to make any changes on the Sprint Goal so that the quality is not decreasing. The Product Backlog can be refined, and the scope can be adjusted. \cite{scrum_guide}

\textbf{Sprint Planning} In the sprint planning the following questions will be answered: Why is this Sprint valuable? What can be done this Sprint? How will the chosen work get done? One of the roles should be able to answer the questions. \cite{scrum_guide}

\textbf{Daily Scrum} should be a 15-minute meeting where every developer provides the current status of his task. It helps to improve communication and helps in decision-making. \cite{scrum_guide}

\textbf{Sprint Review} covers the focus on the outcome of a sprint. The team shows the progress to the stakeholders. \cite{scrum_guide}

\textbf{Sprint Retrospective} helps to improve the quality of a Sprint. The Scrum Team critics what went wrong and find solutions to improve it. \cite{scrum_guide}

\subsection{Scrum Artefacts} \label{sec:Scrum Artefacts}

Artefacts define what work has to be done. To ensure transparency commitments are defined. The commitment for the Product Backlog is the Product Goal, for the Sprint Backlog it is the Sprint Goal and for the Increment it is the definition of 'Done'. Further explanations on the different artefacts are as followed: \cite{scrum_guide}

\textbf{Product Backlog} is a list of items that can be transferred to the Sprint Backlog. The items can be a task or a number of tasks that have to be done. Each item have a 'size' that define the hours which have to be invested to finish the task. \cite{scrum_guide}

\textbf{Sprint Backlog} is a list of items which have to be worked on for the current sprint to accomplish the Sprint Goal. \cite{scrum_guide}

\textbf{Increment} is a step to the Product Goal direction. The increments are named like that because they are additive to previous increments. The Definition of 'Done' defines the state of the Increment so that the requirements of that Increment is met. \cite{scrum_guide}
