% Lorem Ipsum
\usepackage{lipsum}

% Language and Font
\usepackage[utf8]{inputenc}
\usepackage[T1]{fontenc}
\usepackage[ngerman, english]{babel}
\usepackage{lmodern}
\usepackage{xcolor}

% Mathematical Symbols
\usepackage{amssymb}

% Images
\usepackage{graphicx} % for including graphics
\graphicspath{ {img/} } % root directory for graphics
\usepackage{subcaption} % figures inside figures (e.g. for side by side figures)
\usepackage{float} % required for positioning figures with [H] (and lots of other stuff)
\usepackage{wrapfig} % floating figures

% Import PDFs
\usepackage{pdfpages}

\usepackage[headsepline, footsepline, plainfootsepline]{scrlayer-scrpage}
\clearpairofpagestyles
\automark{chapter}
\ihead{\leftmark{}}
\ofoot[\thepage]{\thepage}

% Section Headers
\usepackage{titlesec}
\titleformat{\chapter}[display] % Remove 'Chapter X' from headings
  {\normalfont\bfseries}{}{0pt}{\Huge}
\usepackage{parskip}
\setlength{\parindent}{0cm} % No identitation after new paragraphs
\setlength{\headheight}{14.5pt}
\setlength{\marginparwidth}{2cm}

% Paragraph so fertigmachen, dass er als Titel angezeigt werden kann
\makeatletter
\renewcommand\paragraph{\@startsection{paragraph}{4}{\z@}%
            {-2.5ex\@plus -1ex \@minus -.25ex}%
            {1.25ex \@plus .25ex}%
            {\normalfont\normalsize\itshape\bfseries}}
\makeatother

% Package um Todo Notizen einzufügen und farblich hervorzuheben
\usepackage[ngerman,linecolor=gray,bordercolor=gray,backgroundcolor=yellow]{todonotes}

% Code Highlighting
\usepackage{listings}
\definecolor{mygreen}{rgb}{0,0.6,0}
\definecolor{mymauve}{rgb}{0.58,0,0.82}
\lstset{ %
  basicstyle=\ttfamily,        % size of fonts used for the code
  breaklines=true,                 % automatic line breaking only at whitespace
  captionpos=b,                    % sets the caption-position to bottom
  commentstyle=\color{mygreen},    % comment style
  escapeinside={\%*}{*)},          % if you want to add LaTeX within your code
  keywordstyle=\color{blue},       % keyword style
  stringstyle=\color{mymauve},     % string literal style
  extendedchars=true,
  frame=single,
  frameround=tttt,
  framexleftmargin=1pt,
  framexrightmargin=1pt
}

% Hyperlinks
\usepackage[hidelinks, pdftex]{hyperref}
\hypersetup{%
  pdftitle={Formula Student - Path Planning Algorithm},
% Use \plainauthor for final version.
  pdfauthor={Marco Forster und Dan Hochstrasser},
  pdfdisplaydoctitle=true, % For Accessibility
  bookmarksnumbered,
  pdfstartview={FitH}
}

% Enumerated list modification
\usepackage{enumitem}

% Diagrams
\usepackage{pgfplots}
\pgfplotsset{compat=1.17}
\usepackage{pgfplotstable}
\usepackage{sfmath}
\usetikzlibrary{patterns}

% Entfernt den Blocksatz beim Literaturverzeichnis und formatiert schöner als nur \raggedright
\usepackage{ragged2e}

% Glossaries
\usepackage[style=long,nonumberlist,toc,xindy,acronym,nomain]{glossaries}
\makenoidxglossaries
\loadglsentries{glossary.tex}
