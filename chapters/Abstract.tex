Formula Student is a worldwide engineering competition held in different locations across the globe. The goal of every participating team is to develop a race car from scratch to compete against other teams. The race cars are either powered by a combustion engine or an electric motor which have different subcategories in the competition. The \acrlong{zur} team from the \acrlong{zhaw} built an electric-powered race car with driverless capabilities. Sensors detect the cones on the track, which mark the track's limits, and the car's position. A path planning algorithm needs to maneuver the inside the track. This thesis solves this problem with two different algorithms. The first algorithm, the so-called ``Exploration Algorithm'', calculates the middle line of the track. The Exploration Algorithm works on different tracks like the ``Acceleration'' track (a straight line) and Track Drive tracks that are unknown to the team until race day. Further improvements can be made on the ``Skidpad'' track. A static implementation, where the coordinates of the track are saved manually, can be used to solve the issues on that particular track. The output of the exploration algorithm will be delivered to the second algorithm, the so-called ``Optimization Algorithm'', which optimizes the middle line with the information of the car's attributes to an optimal racing line. The Optimization Algorithm works very well with the Minimum Curvature objective, but it does not perform the best compared to the Minimum Time objective. A switch to the Minimum Time objective can further improve the estimated lap times after all required parameters can be provided. Optimizing the track in parts during the exploration phase can further compensate for the longer computation times needed for the objective.