\begin{itemize}
    \item In der Regel ist zumindest ein kurzes Theoriekapitel notwendig. Es nimmt Bezug auf das thematische Oberthema, aber natürlich nicht auf allgemeine theoretische Grundlagen etwa aus der Naturwissenschaft.
\end{itemize}

\section{Formula SAE Competitions}
Each competition is split into two classes: an \acrlong{cv} (\acrshort{cv}) class and an \acrlong{ev} (\acrshort{ev}) class, which was first introduced in 2010. Additionally, vehicles of both classes can participate in an \acrlong{dc} (\acrshort{dc}). After a series of technical inspections, vehicles will compete in a series of static and dynamic events, and the team with the most overall points will win the competition for its class or the \acrshort{dc} respectively. The cars will be driven on three different tracks: an acceleration track, a skid pad track and a track that is not known to the team. The static discipline has three subcategories: engineering design (150 points), cost (100 points) and business plan (75 points). The dynamic discipline is divided into five subcategories: endurance (325 points), efficiency (100 points), autocross (100 points), skid pad (75 points) and acceleration (75 points). The point system, and how the competition works varies by the competition event itself. The mentioned points are based on the Formula Student Germany competition.
\cite{amz_racing_history}

\subsection{Static Events}

\subsection{Dynamic Events}

\subsection{Formula Student Switzerland}

\subsection{Formula Student Germany}

\subsection{Formula Student Alpe Adria}

\section{Driverless}
\lipsum[1]

\section{Robot Operating System (ROS)}
The Robot Operating System (ROS) is not, like the name may suggest, a full-fledged operating system, but a set of software libraries and tools for the development of robot applications. The open-source robotics middleware comes shipped with capable developer tools, drivers and advanced algorithms. \cite{ros2_documentation}

There are currently two major versions of ROS which are seeing releases, ROS 1 and ROS 2. \cite{ros2_distributions} Beginning with releases after 'Foxy Fitzroy', releases in odd years will be non-LTS (Long Term Support) and will only be supported for 1.5 years, while new releases in even years are going to be long-term supported and will be supported for 5 years. \cite{ros2_releases_and_target_platforms}

The work done in this thesis have been done using the ROS 2 release 'Foxy Fitzroy', released on June 5th, 2020. This release will be supported till the end of May 2023. \cite{ros2_distributions}

\subsection{ROS Graph}
There are 5 main concepts of ROS 2 that make up the ROS (2) graph:
\begin{enumerate}
    \item Nodes
    \item Topics
    \item Services
    \item Parameters
    \item Actions
\end{enumerate}

The ROS graph is a network of ROS 2 elements which processes data simultaneously. The graph encompasses all executables and the connections between them.

\begin{figure}[H]
    \centering
    \includegraphics[width=\columnwidth]{ROS2_Main_Concepts.png}
    \caption{The five main concepts of ROS 2 pictured as a network of nodes.}
    \label{fig:ROS 2 main concepts}
\end{figure}

\subsubsection{Nodes}
A node is a fundamental ROS 2 element that serves a single, modular purpose in a robotics system. % https://docs.ros.org/en/foxy/Tutorials/Understanding-ROS2-Nodes.html

\subsubsection{Topics}
 Nodes publish information over topics, which allows any number of other nodes to subscribe to and access that information. % https://docs.ros.org/en/foxy/Tutorials/Topics/Understanding-ROS2-Topics.html

\subsubsection{Services}
Services are based on a call-and-response model, versus topics’ publisher-subscriber model. Services only provide data when they are specifically called by a client. % https://docs.ros.org/en/foxy/Tutorials/Services/Understanding-ROS2-Services.html

\subsubsection{Parameters}
Nodes have parameters to define their default configuration values. %https://docs.ros.org/en/foxy/Tutorials/Parameters/Understanding-ROS2-Parameters.html

\subsubsection{Actions}
Actions are built on topics and services and consist of a goal, feedback, and a result. Actions are like services that allow you to execute long-running tasks, provide regular feedback, and are cancellable. % https://docs.ros.org/en/foxy/Tutorials/Understanding-ROS2-Actions.html

\section{Path Planning Algorithms}
Path planning algorithms are used in many environments for example in assembling a car with a robot arm, parking a car autonomously or finding a landmine with a robot in a military operation. 

\subsection{Overview}
A path can be if a human goes from A to B and follows certain signs for example when hiking on a mountain. Otherwise, a human can follow a path that is generated on a GPS system in a car. The difference lies by a manual plan versus a generated plan by a machine. This section addresses only the generated one.

The planning consists of a state, time that is needed, actions, initial / goal state, criterion and a plan. The state is normally represented by a planning algorithm. Time is based on a sequence of decision that has to be made in a certain time. Actions manipulate the state and cover how the state should be changed. The initial and goal state defines where the planner should start and finish.
The criterion defines the outcome and the boundaries of the planner. The type of the criterion is either based on the feasibility or optimality. Feasibility covers the arrival of the goal state without the concern of efficiency. Optimality finds a plan with an optimized performance in mind. The plan in general covers the strategy or behaviour on a decision maker.
\cite{planning_algorithms_steven_m_lavalle}

\subsection{Algorithms, Planners and Plans}
An algorithm usually consists of a machine and environment where there are actuations from the machine to the environment and sensing from the environment to the machine. The problem in planning is often that the machine interacts with the physical world which has uncountable different influences on the plan that has to be calculated. This is why the environment is often an approximation of the real world since not every sensing mechanism from the environment to the machine can be processed.

Planners are constructing plans. They are either a machine or a human. When the planner is a machine then it is usually a planning algorithm. If the planner is a human then the human itself is the algorithm which makes the decision.

The plans can be used in three different ways: Execution, refinement or hierarchical inclusion. Execution runs the plan on a simulation or a robot which is connected to the real world. Refinement tries to find a better plan. The hierarchical inclusion gives the plan to a higher level plan that is consisting of sub plans.

\cite{planning_algorithms_steven_m_lavalle}

\subsection{Path Planning Algorithm Categories}
A path planning algorithm can be categorized in one of the following groups:
\begin{itemize}
    \item Motion Planning
    \item Decision-Theoretic Planning
    \item Planning Under Differential Constraints
\end{itemize}

Motion planning is interested in planning in a continuous state spaces. This means that the space is not static and changes in relation to time. 


\section{Hardware}

\subsection{Nvidia Jetson}
\lipsum[1]

\subsection{Lidar}
\lipsum[1]

\subsection{Stereo Camera}
\lipsum[1]

\section{Languages and Tools}

\begin{itemize}
    \item ROS 2
    \item Python
    \item LaTeX
    \item Git
    \item VS Code
    \item Azure DevOps
    \item GitHub
    \item Simulation tools
    \item other tools
\end{itemize}

\lipsum[1]