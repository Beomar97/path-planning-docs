Formula Student ist ein weltweiter Ingenieur Wettbewerb, welcher an verschiedenen Standorten auf der ganzen Welt ausgetragen wird. Das Ziel jedes Teams ist es, ein Rennauto von neu auf zu entwickeln und mit diesem am Renntag gegen andere Teams anzutreten. Es gibt Verbrennungs- und Elektromotor angetriebene Rennautos, welche in ihren eigenen Kategorien gegeneinander antreten. Das \acrlong{zur} Team der Zürcher Hochschule für angewandte Wissenschaften hat ein elektrogetriebenes Rennauto mit fahrerlosen Kapazitäten entwickelt. Es existieren Disziplinen, wo ein Rennfahrer das Fahrzeug steuert und welche ohne einen, wobei sich das Fahrzeug dann autonom von selbst fortbewegen muss. Zwei von drei Strecken, welche abgefahren werden müssen, sind im Voraus bekannt, während die letzte Strecke bis zum Renntag unbekannt bleibt. Mit den erfassten Pylonen am Rande der Strecke, welche die Strecke begrenzen, und der Position des Rennautos kommt ein Pfadplanungsalgorithmus zum Zuge, welches dem Rennauto erlaubt, zwischen den Pylonen zu manövrieren. Diese wissenschaftliche Arbeit gibt zwei Algorithmen her, welche dieses Problem in Kombination lösen. Der erste Algorithmus, der sogenannte Erkundungsalgorithmus, berechnet die Mittellinie der Strecke und übergibt diese dem zweiten Algorithmus, dem Optimierungsalgorithmus. Der zweite Algorithmus optimiert die Mittellinie zu einer Ideallinie. Dies geschieht zusätzlich mit der Beeinflussung der Autoeckdaten. Der Erkundungsalgorithmus kann für die sogenannte ``Acceleration'' Strecke (welche eine Gerade abbildet) und für unbekannte Strecken angewendet werden. Eine Erweiterung des Erkundungsalgorithmus für die ``Skidpad'' Strecke müsste noch realisiert werden. Die ``Skidpad'' Strecke ist eine Strecke, welche eine liegende Acht repräsentiert, bei der das Rennauto zweimal auf dem rechten Kreis und zweimal auf dem linken Kreis fahren soll, bevor es anschliessend ins Ziel manövriert werden muss. Dieses Problem kann mit einer statischen Implementation, welches die vorausgespeicherten Koordinaten der Mittellinie nutzt, gelöst werden.
